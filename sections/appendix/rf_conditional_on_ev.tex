\section{Proof of Conditionals in Theorem~\ref{thm: conditional distributions of rf}}
\label{sec: proof of conditionals}

In this section we want to determine the distribution of \(\rf\mid
\Lambda\). As a warmup we will first consider
\(\E[Z\mid \shift]\). \(\shift\) is the shift we can obtain from \(\Lambda_i\) by
averaging if we have sufficiently many. But we can not really separate
\(\epsilon\) from \(\rf\) using only \(\Lambda\), so \(\rf\mid \shift\) will be
close to \(\rf\mid\Lambda\) for large
\(\dimension\).

\subsection{\texorpdfstring{\(\rf\mid \shift\)}{Z|H0}}

Since \(\rf\) and \(\shift\) are multivariate normal, we have 
\[
	\rf\mid \shift\sim \normal(\E[\rf\mid \shift], \Var(\rf\mid \shift)).
\]
So we only need to calculate the conditional expectation \(\E[\rf\mid \shift]\) and
the remaining variance
\[
	\Var(\rf\mid \shift)\overset{\text{def}}
	=\E[(\rf-\E[\rf\mid \shift])^2\mid \shift].
\]
For the conditional expectation it is sufficient to calculate the best linear
unbiased estimator (BLUE) as it coincides with the conditional expectation in
the Gaussian case.

Let \(\langle X, Y\rangle := \E[XY]\) and \(\|\cdot\|^2\) the induced norm.
Then we want to find the \(\alpha\in\real\) which minimizes
\[
	\|Z-\alpha \shift\|^2
	= (1-\tfrac{\sqC'(0)}{\sqC(0)}\alpha)^2 \underbrace{\|\rf\|^2}_{=\sqC(0)}
	+ \alpha^2 \underbrace{\|\epsilon\|^2}_{=\sqC''(0)\mathrlap{-\frac{\sqC'(0)^2}{\sqC(0)}}}
\]
The first order condition is therefore
\[
	0\overset!= -2 \tfrac{\sqC'(0)}{\cancel{\sqC(0)}}
	(1-\bcancel{\tfrac{\sqC'(0)}{\sqC(0)}\alpha})\cancel{{\sqC(0)}} + 2\alpha
	\left(\sqC''(0)-\bcancel{\tfrac{\sqC'(0)^2}{\sqC(0)}}\right),
\]
which leads to \(\alpha = \frac{\sqC'(0)}{\sqC''(0)}\) and therefore
\[
	\E[\rf\mid \shift] = \tfrac{\sqC'(0)}{\sqC''(0)} \shift.
\]
Using Lemma~\ref{lem: conditional variance in the gaussian case} we finally
have
\[\begin{aligned}
	\Var(\rf\mid \shift)
	\overset{\ref{lem: conditional variance in the gaussian case}}&=
	\Var(\rf) - \Var(\E[\rf\mid \shift])
	= \sqC(0) - \tfrac{\sqC'(0)^2}{\sqC''(0)^2}\underbrace{\Var(\shift)}_{
		= \Var(\epsilon) \mathrlap{+ \frac{\sqC'(0)^2}{\sqC(0)^2}\Var(\rf)}
	}\\
	&= \sqC(0) - \tfrac{\sqC'(0)^2}{\sqC''(0)^2}\left[
		\sqC''(0) - \cancel{\tfrac{\sqC'(0)^2}{\sqC(0)} +  \tfrac{\sqC'(0)^2}{\sqC(0)}}
	\right]\\
	&=\sqC(0) - \tfrac{\sqC'(0)^2}{\sqC''(0)}.
\end{aligned}\]
In summary we have:
\[
	\rf\mid \shift \sim \normal\left(
		\tfrac{\sqC'(0)}{\sqC''(0)} \shift,
		\sqC(0) - \tfrac{\sqC'(0)^2}{\sqC''(0)}
	\right)
\]

\subsection{\texorpdfstring{\(\rf\mid H_1,\dots,H_\dimension\)}{Z|H1,...,HN}}

As another intermediate step, we are going to calculate the distribution of
\(\rf\mid H_1,\dots,H_\dimension\) using the diagonal elements
\(H_1,\dots,H_\dimension\) of \(H\). We will later find out that it is identical
to the one we want anyway.

Using the same approach via the BLUE, we want to minimize
\[
	\begin{aligned}
	&\|\rf - \sum_{i=1}^\dimension\alpha_i H_i \|^2\\
	&= \left(1- \frac{\sqC'(0)}{\sqC(0)}\sum_{i=1}^\dimension \alpha_i\right)^2
	\underbrace{\|\rf\|^2}_{\sqC(0)}
	+  \left(\sum_{i=1}^\dimension \alpha_i\right)^2 \underbrace{\|\epsilon\|^2}_{
		=\sqC''(0)-\tfrac{\sqC'(0)^2}{\sqC(0)}
	}
	+ \sum_{i=1}^\dimension \alpha_i^2 \underbrace{\|\tilde{H}_i\|^2}_{2\sqC''(0)}.
	\end{aligned}
\]
We are going to minimize this term in two steps.
\begin{enumerate}
	\item Fix \(\sum_{i=1}^\dimension \alpha_i = s\) and minimize over
	\(\alpha_i\) with this constraint,
	\item minimize the result over \(s\).
\end{enumerate}
In the first part, everything is constant except for the last part. So
we need to find the \(\alpha_i\) which minimize
\[
	\sum_{i=1}^\dimension \alpha_i^2
\]
under the given constraint \(\sum_{i=1}^\dimension \alpha_i = s\). As it turns
out
\[
	1 = \left(\sum_{i=1}^\dimension \frac{\alpha_i}s\right)^2
	= \left(\frac1\dimension\sum_{i=1}^\dimension \frac\dimension{s}\alpha_i\right)^2
	\overset{(\cdot)^2 \text{ convex.}}\le \frac1\dimension \sum_{i=1}^\dimension 
	\left(\frac\dimension{s}\alpha_i\right)^2
	= \frac\dimension{s^2} \sum_{i=1}^\dimension \alpha_i,
\]
implies
\[
	\sum_{i=1}^\dimension \left(\frac{s}\dimension\right)^2 = \frac{s^2}\dimension
	\le \sum_{i=1}^\dimension \alpha_i.
\]
So \(\alpha_i=\frac{s}\dimension\) minimizes our constraint optimization problem.
Inserting this solution into our problem results in
\[
	\left(1- \tfrac{\sqC'(0)}{\sqC(0)}s\right)^2
	\sqC(0)
	+  s^2 \left(
		\sqC''(0)-\tfrac{\sqC'(0)^2}{\sqC(0)}
	\right)
	+ 2\frac{s^2}\dimension \sqC''(0).
\]
Except for the last summand this is the same optimization problem as before.
The first order condition results in
\[
	s = \frac{\sqC'(0)}{\sqC''(0)(1+\tfrac4{\dimension})}
\]
So we finally get
\begin{align*}
	\E[\rf\mid H_1,\dots,H_\dimension]
	&= \frac{\sqC'(0)}{\sqC''(0)(1+\tfrac4{\dimension})}
	\frac1\dimension\underbrace{\sum_{i=1}^\dimension H_i}_{
		=\trace(H) \mathrlap{=\sum_{i=1}^\dimension\Lambda_i}
	}\\
	&= \frac{\sqC'(0)}{\sqC''(0)(1+\tfrac4{\dimension})}
	\bar{\Lambda}
\end{align*}
where \(\bar{\Lambda}\) is the mean eigenvalue. Lastly we want to determine
the explained variance
\begin{align*}
	\Var(\E[\rf\mid H_1,\dots,H_\dimension])
	&= \left(\frac{\sqC'(0)}{\sqC''(0)(1+\tfrac4{\dimension})}\right)^2
	\Var\left(
		\epsilon + \tfrac{\sqC'(0)}{\sqC(0)}\rf + \frac{1}{\dimension}\sum_{i=1}^\dimension \tilde{H_i}
	\right)\\
	&= \frac{\sqC'(0)^2}{\sqC''(0)^2(1+\tfrac4{\dimension})^2}
	\left(
		\Var(\epsilon) + \tfrac{\sqC'(0)^2}{\sqC(0)} + \tfrac1\dimension\Var(\tilde{H}_1)
	\right)\\
	&= \frac{\sqC'(0)^2}{\sqC''(0)^2(1+\tfrac4{\dimension})^2}
	(\sqC''(0) + \tfrac2\dimension\sqC''(0))\\
	&= \frac{\sqC'(0)^2(1+\tfrac2\dimension)}{\sqC''(0)(1+\tfrac4{\dimension})^2}.
\end{align*}
Since the off-diagonal elements are independent from \(\rf\), we get with
Lemma~\ref{lem: conditional variance in the gaussian case}
\[
	\rf\mid H = \rf \mid H_1,\dots, H_\dimension \sim \normal\left(
		\frac{\sqC'(0)}{\sqC''(0)(1+\tfrac4{\dimension})}\bar{\Lambda},
		\sqC(0)-\frac{\sqC'(0)^2(1+\tfrac2\dimension)}{\sqC''(0)(1+\tfrac4{\dimension})^2}
	\right).
\]

\subsection{\texorpdfstring{\(\rf\mid \Lambda\)}{Z|Λ}}

So we know that \(\E[f(\rf)\mid H]\) is \(\Lambda\)
measurable. And since \(\Lambda\) are as eigenvalues
of \(H\) \(\sigma(H)\)-measurable, we have \(\sigma(H)\supseteq \sigma(\Lambda)\)
and therefore get
\[
	\E[f(Z)\mid \Lambda]
	= \E[\E[f(Z)\mid H] \mid \Lambda]
	= \E[f(Z)\mid H].
\]
Which finally implies \(\rf\mid H = \rf\mid \Lambda\).