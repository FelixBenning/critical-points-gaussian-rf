\section{Counting Critical Points of Random Fields}

\subsection{Random Fields}

\begin{definition}[Random Field]
	A collection of random variables \((\rf(t))_{t\in\real^\dimension}\) is called
	\textbf{random field} over \(\real^\dimension\). The \textbf{covariance function}
	is defined as
	\begin{equation*}
		\C(x,y) = \Cov(\rf(x), \rf(y)).
	\end{equation*}
	A random field is called \textbf{Gaussian random field}, if all finite
	dimensional marginals are multivariate Gaussian.
	A random field is called \textbf{centered}, if for all \(t\in\real^\dimension\)
	\[
		\E[\rf(t)] = 0.
	\]
	A random field is called \textbf{strictly stationary}, if for all
	\(h\in\real^\dimension\) we the following equality in distribution
	\[
		(\rf(t+h))_{t\in\real^\dimension} \overset{d}= (\rf(t))_{t\in\real^\dimension}.
	\]
	In particular this implies	\textbf{(weak) stationarity}, i.e.
	\begin{equation*}
		\C(x,y) = \C(x-y)\qquad \text{and} \qquad \E[\rf(t)]=\mu.
	\end{equation*}
	For centered Gaussian random fields, strict and weak stationarity coincides.
	A random field is called \textbf{isotropic} (rotation invariant) if	
	\begin{equation*}
		\C(x,y) = \sqC\left(\frac{\|x-y\|^2}{2}\right).
	\end{equation*}
\end{definition}

\begin{lemma}[Covariance and Derivative]
	\label{lem: covariance of derivative}
	We have
	\begin{align*}
		\Cov(\partial_i\rf(x), \rf(y)) &= \partial_{x_i} \C(x,y)\\
		\Cov(\partial_i\rf(x), \partial_j \rf(y)) &= \partial_{x_i y_j} \C(x,y)
	\end{align*}
\end{lemma}
\begin{proof}
	\fxwarning{missing proof}	
\end{proof}
\begin{remark}\label{rem: covariance uncorrelated}
	For stationary random fields \(\C(x,y)=\C(x-y)\), the covariance
	function \(\C\) is symmetric, i.e. \(\C(h)=\C(-h)\)
	and therefore the derivative antisymmetric, i.e.
	\(\partial_i\C(-h)=-\partial_i\C(h)\). In particular
	\begin{equation*}
		\Cov(\nabla\rf(x), \rf(x)) = \nabla \C(0)=0.
	\end{equation*}
\end{remark}
\begin{corollary}[Gaussian Case] \label{cor: uncorr leads to indep in gaussian case}
	If \(\rf\) is a stationary gaussian random field, \(\rf(x)\) and
	\(\nabla\rf(x)\) are independent multivariate gaussian for every \(x\).
\end{corollary}

\subsection{Getting to the Point Process}

The set of critical points of a random field, is surprise, surprise: random.
So how would one formalize a random set in probability theory? It turns
out that this becomes much easier if we translate our random set to a random
measure. Let us consider the random set \(\Phi\)
\[
	\Phi = \{ X_1, \dots, X_K\}
\]
where \(X_1,\dots, X_K\in\real^\dimension\) are the random points of this set. Then with some
abuse of notation, we can define the measure
\[
	\Phi :
	\begin{cases}
		\borel(\real^\dimension) \to \real \\
		A \mapsto \Phi(A):= \sum_{X\in \Phi}\delta_X(A) 
	\end{cases}
\]
Where \(\delta_X(A)=\ind_A(X)\) is the dirac measure. This measure \(\Phi\)
essentially counts the number of points in the intersection \(\Phi\cap A\).
Testing with different \(A\) allows us to fully determine, what
the original points in \(\Phi\) were. So in this sense, the measure is
equivalent to the set. And it turns out that it is much more convenient
formally, to define random measures, than it is to define random sets.
So let us introduce some nomenclature for measures.

A measure \(\mu\) is called
\begin{itemize}
	\item \textbf{locally finite}, if \(\mu(A) < \infty\) for all bounded
	\(A\in\borel(\real^\dimension)\),
	\item \textbf{counting measure} if it is discrete,
	\item \textbf{simple}, if \(\mu(\{x\})\in\{0,1\}\).
\end{itemize}
Let \(\locFiniteMeasure\) be the \textbf{set of all locally finite counting measures}
on \(\real^\dimension\) and \(\locFiniteMeasAlg\) the smallest \(\sigma\)-algebra
such that
\[
	\pi_B : \begin{cases}
		\locFiniteMeasure \to \borel(\real)\\
		\mu \mapsto \mu(B)
	\end{cases}
\]
is measurable for every \(B\in\borel(\real^\dimension)\). I.e.
\[
	\locFiniteMeasAlg := \sigma(\pi_B : B\in\borel(\real^\dimension))
\]

\begin{definition}[Point Process]
	A \textbf{point process} is a measurable mapping \(\Phi:\Omega\to\locFiniteMeasure\)
	from a probability space \((\Omega, \mathcal{A}, \Pr)\) into the set of
	locally finite counting measures. Due to the definition of \(\locFiniteMeasAlg\)
	\[
		\Phi(B) := \pi_B\circ \Phi : \Omega \to \real
	\]
	is a random variable for any \(B\in\borel(\real^\dimension)\). If
	\[
		\Lambda:\begin{cases}
			\borel(\real^\dimension)\to [0,\infty)\\
			B \mapsto \E[\Phi(B)]
		\end{cases}
	\]
	is locally finite, then it is called the \textbf{intensity measure}
	(expectation measure) of \(\Phi\). The intensity measure essentially tells
	us, how many points we should expect in some set \(B\) on average.
	If \(\Lambda(dx) = \lambda dx\) for some
	constant \(\lambda\in\real\), then \(\lambda\) is called the
	\textbf{intensity} of \(\Phi\).
	A point process is called \textbf{simple}, if \(\Phi(\omega)\) is simple
	for every \(\omega\in\Omega\).
\end{definition}

Often our random set of points has a certain structure. E.g. we have
random points in space \(X_1,\dots, X_K\) with marks \(M_1,\dots, M_k\) which
might for example represent the value of some random field \(\rf\) at our
points in space, i.e. \(M_i=\rf(X_i)\). Then our random graph 
\[
	\Phi = \{ (X_1, M_1), \dots, (X_K, M_K) \}
\]
is just a point process in higher dimensional space, with points \((X_i, M_i)\).
But of course not every point process can be interpreted as a point with a mark.
So we define the special case

\begin{definition}[Marked Point Process]
	A point process \(\Phi\) on \(\locFiniteMeasure[\dimension+n]\) is called
	\textbf{marked point process}, if \(\Phi_p = \Phi(\cdot \times \real^n)\)
	is a point process on \(\locFiniteMeasure\).

	A marked point process is called \textbf{stationary}, if for all
	\(h\in\real^\dimension\), \(A_i\in\borel(\real^\dimension)\),
	\(M_i\in\borel(\real^n)\) and \(B_i\subseteq\nat_0\) we have
	\[
		\Pr(\Phi(A_i\times M_i) \in B_i, i=1,2,\dots)
		= \Pr(\Phi(A_i+h \times M_i)\in B_i, i=1,2,\dots)
	\]
\end{definition}

Notice that for a marked point process to be stationary, we only shift the
spacial part of the point. This stands in contrast to a stationarity definition
for ordinary point processes. This is the main reason we introduce this
additional machinery. In our case we are interested in the height (mark) of
our random field at our critical points. And the following Lemma will allow us
to recover the distribution of marks on these critical points

\begin{lemma}[Campbell]\label{lem: Campbell}
	For a stationary marked point process \(\Phi\) with intensity \(\Lambda\),
	we have	
	\[
		\Lambda(A \times L) = \lambda(L) |A|
	\]
	where \(|A|\) is the lebesgue measure of \(A\). We call
	\[
		\Pr_M(L) = \frac{\lambda(L)}{\lambda(\real^n)}
	\]
	the \textbf{distribution of marks}, because we have for integrable \(f\)
	(Campbell)
	\[
		\E\left[\int f(x,m) \Phi(dx,dm)\right]
		= \lambda \int \int f(x,m) \Pr_M(dm) dx
	\]
\end{lemma}
\begin{proof}
	\fxwarning{TODO: Fakt in 3.5 räumliche statistik skript schlather}
\end{proof}

\subsection{Making it Count}

\begin{definition}[Counting Critical Points]
	Let \(\rf\) be a random field. We implicitly\footnote{
		\(\borel(\real^\dimension)\times\borel(\real^{1+\dimension^2})\) is a semiring. So by
		Carathéodory's extension theorem there is a unique extension of
		\(\crit\) to
		\(\sigma(\borel(\real^\dimension)\times\borel(\real^{1+\dimension^2}))=\borel(\real^{\dimension+1+\dimension^2})\).
		Use this extension to obtain \(\crit(\omega)\in\locFiniteMeasure[\dimension+1+\dimension^2]\).
	} define the marked point process of critical points as
	\begin{align*}
		\crit:
		\begin{cases}
			\borel(\real^\dimension)\times\borel(\real^{1+\dimension^2}) \to \real\\
			(\Vol, M) \mapsto
			\#\left\{t\in\Vol:
				\nabla\rf(t)=0,
				(\rf(t), \nabla^2 \rf(t))\in M
			\right\}.
		\end{cases}
	\end{align*}
	This is a marked point process with \((\rf(t), \nabla^2\rf(t))\) as marks.
	Let \(\negEV(H)\) be the number of negative eigenvalues\footnote{
		Since the eigenvalues are continuous in the entries of the matrix, the
		number of negative eigenvalues is a measurable function.
	} of \(H\in\real^{\dimension\times\dimension}\). Then, the number of minima in
	\(\Vol\) with height \(\rf(t)\) in \(A\) can be expressed as
	\[
		\crit^0(\Vol, A):= \crit(\Vol, A\times\negEV^{-1}(0)).
	\]
	More generally we define the point-process of critical points with index
	\(\alpha\) (the share of negative eigenvalues) as
	\[
		\crit^\alpha(\Vol, A)
		:= \crit(\Vol, A\times\negEV^{-1}(\dimension\alpha)).
	\]
\end{definition}
\begin{proof}[Proof (Measurability)]
	First note that	
	\begin{align*}
		\locFiniteMeasAlg[\dimension+1+\dimension^2]
		&= \sigma(\pi_B : B\in\borel(\real^{\dimension+1+\dimension^2}))\\
		&=\sigma(\underbrace{
			\{\pi_B^{-1}(C) : B\in\borel(\real^{\dimension+1+\dimension^2}), C\in\borel(\real)\}
		}_{=:\mathcal{E}}
		)
	\end{align*}
	since it is sufficient to prove measurability on the generator
	\(\mathcal{E}\), we only need that the following sets are measurable
	\begin{align*}
		(\crit)^{-1}(\pi^{-1}_{B}(C))
		&= (\pi_{B} \circ \crit)^{-1}(C)
		= (\crit(B))^{-1}(C)\\
		&= \#\left\{\left(t,\rf(t),\nabla^2\rf(t)\right) \in B:
			\nabla\rf(t)=0
		\right\}\\
		&= \lim_{\epsilon\to 0} \sum_{k\in I^\epsilon}
		\ind\left\{
			\inf_{t\in \left(\text{id}, \rf, \nabla^2\rf\right)^{-1}(B^\epsilon_k) }
		|\nabla \rf(t)| = 0
		\right\}
	\end{align*}
	\fxwarning{probably not quite right yet}
	where \((B^\epsilon_k)_{k\in\nat}\) is an \(\epsilon\) tiling of \(B\) (countable).
	We can swap the limit with the series due to monotone convergence.
\end{proof}

\begin{lemma}[Stationarity]
	If \(\rf\) is strictly stationary, \(\crit\) is stationary.
\end{lemma}
\begin{proof}
	Let \(h\in \real^\dimension\) and note that
	\[
		\tilde{\rf}(t) := \rf(t+h)
	\]
	has the same distribution as \(\rf\) since \(\rf\) is strictly stationary. Define
	\(\tilde{\crit}\) using \(\tilde{\rf}\) in place of \(\rf\). Then
	\(\tilde{\crit}\) has the same distribution as \(\crit\).
	And we have
	\begin{align*}
		&\crit(\Vol+h, M)\\
		&= \#\left\{t\in\Vol+h:
			\nabla\rf(t)=0,
			(\rf(t),\nabla^2\rf(t))\in M
		\right\}\\
		&= \#\left\{t\in\Vol:
			\nabla\tilde{\rf}(t)=0,
			(\tilde{\rf}(t),\nabla^2\tilde{\rf}(t))\in M
		\right\}\\
		&= \tilde{\crit}(\Vol, M).
	\end{align*}
	So \(\crit\) is stationary:
	\begin{align*}
		\Pr\left(\crit((\Vol_i+h)\times M_i)\in B_i, i=1,2,\dots\right)
		&= \Pr\left(\tilde{\crit}(\Vol_i\times M_i)\in B_i, i=1,2,\dots\right)\\
		&= \Pr\left(\crit(\Vol_i\times M_i)\in B_i, i=1,2,\dots\right).
		\qedhere
	\end{align*}
\end{proof}

To count critical points with restrictions on the second derivative, the second
derivative has to exist. For that, note that
\[
	\partial_{x_i x_j, y_i y_j}\C(x,y) = \Cov(\partial_{ij}\rf(x), \partial_{ij}(y)),
\]
and it turns out that existence of the term on the left essentially ensures
existence of \(\partial_{ij}\rf\)\fxnote{reference}. For continuous second
derivatives, we want a little bit more \parencite[cf.
Theorem~1.4.1]{adlerRandomFieldsGeometry2007}. In fact, we want the following:

\begin{assumption}[Smooth Random Field]\label{assmpt: smoothness assumption}
	A sufficient condition for smooth second derivatives \(\partial_{ij}	\rf\)
	of a random field, is
	\begin{equation}\label{eq: sufficient for continuous second derivative}
		\max_{i,j}\E[|\partial_{ij}\rf(t) - \partial_{ij}\rf(s)|^2]
		\le K |\ln(|t-s|)|^{-(1+\alpha)}
	\end{equation}
	for some \(K>0\), \(\alpha>0\) for all \(|t-s|\) small enough.
	In the stationary case this condition simplifies to
	\[
		\max_{i,j}|\partial_{ii jj}\C(0)
		-\partial_{ii jj}\C(h)| \le \frac{K}2 |\ln(|h|)|^{-(1+\alpha)}
	\]
	for \(h\in\real^\dimension\) with \(|h|\) small enough.
\end{assumption}


\begin{theorem}[Kac-Rice Formula]
	\label{thm: kac-rice formula}
	For a centered Gaussian random field \(\rf\) satisfying our smoothness
	Assumption~\ref{assmpt: smoothness assumption}, the intensity measure of the
	critical points \(\crit\) is given by
	\begin{align*}
		&\Lambda(\Vol,M)
		=\E[\crit(\Vol, M)]\\
		&= \int_\Vol \E\left[
			\left|\det(\nabla^2 \rf(t))\right|
			\ind_M(\rf(t), \nabla^2\rf(t))
			\bigm| \nabla\rf(t) = 0 
		\right] \density_{\nabla\rf(t)}(0)dt
		\\
		\overset{\text{stationary}}&=
		|\Vol|\ \underbrace{\E\left[
			|\det(\nabla^2 \rf(0))|
			\ind_M(\rf(0), \nabla^2\rf(0))
		\right] \density_{\nabla\rf(0)}(0)}_{=:\lambda(M)}
	\end{align*}
	where \(\density_{\nabla\rf(t)}\) is the density of \(\nabla\rf(t)\). The
	second equality is only true when \(\rf\) is stationary.
\end{theorem}

\begin{proof}
	This is an immediate result from Corollary~\ref{cor: gaussian kac-rice}
	\parencite[Corollary~11.2.2]{adlerRandomFieldsGeometry2007}. More precisely
	we define
	\begin{align*}
		f(t) &:= \nabla \rf(t)\\
		g(t) &:= (\rf(t), \nabla^2 \rf(t))
	\end{align*}
	We now simply need to check that the covariance function of the components of
	\(\partial_j f^i(t) = \partial_{ji} \rf(t)\) and \(g^i\) (which are
	also just the second derivatives and \(\rf\) itself) satisfy the smoothness
	requirement of the corollary. As smoothness of the second derivative implies
	smoothness of \(\rf\), we only need to check the requirement for \(\partial_{ji} \rf\).
	In other words, we need
	\[
		\max_{i,j}|\partial_{x_i x_j, y_i y_j}\C(t,t)
		+ \partial_{x_i x_j, y_i y_j}\C(s,s)
		-2\partial_{x_i x_j, y_i y_j}\C(s,t)| \le K |\ln(|t-s|)|^{-(1+\alpha)}
	\]
	But that is \eqref{eq: sufficient for continuous second derivative} in
	Assumption~\ref{assmpt: smoothness assumption} written in terms of
	covariances. By application
	of Corollary~\ref{cor: gaussian kac-rice}, we therefore get
	\begin{align*}
		\Lambda(\Vol, M) 
		&= \E[\crit^\alpha(\Vol, M)]
		= \E[\level(f, g: \Vol, M)]\\
		&= \int_\Vol \E\left[
			|\det \nabla f(t)| \ind_M(\rf(t), \nabla^2\rf(t)) \mid f(t) = 0
		\right] \density_{\nabla\rf(t)}(0)dt.
	\end{align*}
	
	
	The second inequality follows as \(\nabla\rf(t)\) is independent of
	\((\rf(t),\nabla^2\rf(t))\) due to Corollary~\ref{cor: uncorr leads to indep
	in gaussian case} which turns the conditional expectation into a normal
	expectation. Lastly replace all \(t\) with \(0\) due to stationarity.
\end{proof}
\begin{remark}
	As can be seen in Theorem~\ref{thm: general kac-rice} \parencite[Theorem
	11.2.1]{adlerRandomFieldsGeometry2007}, \(\rf\) does not necessarily
	need to be Gaussian. We just need it and its derivatives to satisfy a
	list of properties which are satisfied for Gaussian process with
	smooth enough covariance functions.
\end{remark}

\begin{corollary}
	Let \(p(z,h)\) be the joint density of \((\rf(0),\nabla^2\rf(0))\) and define
	\[
		\Omega(z,h) = c |det(h)| p(z,h)
	\]
	with \(c\in\real\) such that \(\int \Omega(z,h)dz dh = 1\).
	If we take a random critical point \(T\) picked uniformly from the set of
	critical points \(\{t\in\Vol : \nabla\rf(t)=0\}\), then the distribution over
	its marks is given by
	\[
		\Pr_{\rf(T),\nabla^2\rf(T)}(A) = \int_A \Omega(z,h) dz dh.
	\]
\end{corollary}
\begin{remark}
	Notice how this distribution of marks is different from
	\[
		\begin{aligned}
		\Pr((\rf(t),\nabla^2\rf(t))\in A \mid \nabla\rf(t) = 0)
		\overset{\text{stationary}}&= \Pr((\rf(0),\nabla^2\rf(0))\in A)\\
		&= \int_A p(z,h)dz dh.
		\end{aligned}
	\]
	In particular the density of marks is \(\Omega\) not \(p\), like it is if
	we simply condition on the fact that \(t\) is a critical point.
\end{remark}
\begin{proof}
	With Lemma~\ref{lem: Campbell}, we have that the distribution of marks is
	given by
	\begin{align*}
		\Pr_{\rf(T),\nabla^2\rf(T)}(A)
		&= \frac{\lambda(A)}{\lambda(\real)}
		\overset{\ref{thm: kac-rice formula}}= \frac{
			\E[|\det(\nabla^2 \rf(0))|
			\ind_A(\rf(0), \nabla^2\rf(0))
			]
		}{
			\E[|\det(\nabla^2 \rf(0))|]
		}\\
		&= \int_A \Omega(z,h) dz dh.
		\qedhere
	\end{align*}
\end{proof}