\section{Counting Critical Points of Random Fields}

\subsection{Random Fields}

\begin{definition}[Random Field]
	A collection of random variables \((\rf(t))_{t\in\real^\dimension}\) is called
	\textbf{random field} over \(\real^\dimension\). The \textbf{covariance function}
	is defined as
	\begin{equation*}
		\C(x,y) = \Cov(\rf(x), \rf(y)).
	\end{equation*}
	A random field is called \textbf{stationary}, if
	\begin{equation*}
		\C(x,y) = \C(x-y).
	\end{equation*}
	It is called \textbf{isotropic} (rotation invariant) if	
	\begin{equation*}
		\C(x,y) = \dimension \sqC\left(\frac{\|x-y\|^2}{2\dimension}\right).
	\end{equation*}
	A random field is called \textbf{Gaussian random field}, if all finite
	dimensional marginals are multivariate Gaussian.
\end{definition}

\begin{lemma}[Covariance and Derivative]
	\label{lem: covariance of derivative}
	We have
	\begin{align*}
		\Cov(\partial_i\rf(x), \rf(y)) &= \partial_{x_i} \C(x,y)\\
		\Cov(\partial_i\rf(x), \partial_j \rf(y)) &= \partial_{x_i y_j} \C(x,y)
	\end{align*}
\end{lemma}
\begin{proof}
	\fxwarning{missing proof}	
\end{proof}
\begin{remark}\label{rem: covariance uncorrelated}
	For stationary random fields \(\C(x,y)=\C(x-y)\), the covariance
	function \(\C\) is symmetric, i.e. \(\C(h)=\C(-h)\)
	and therefore the derivative antisymmetric, i.e.
	\(\partial_i\C(-h)=-\partial_i\C(h)\). In particular
	\begin{equation*}
		\Cov(\nabla\rf(x), \rf(x)) = \nabla \C(0)=0.
	\end{equation*}
\end{remark}
\begin{corollary}[Gaussian Case] \label{cor: uncorr leads to indep in gaussian case}
	If \(\rf\) is a stationary gaussian random field, \(\rf(x)\) and
	\(\nabla\rf(x)\) are independent multivariate gaussian for every \(x\).
\end{corollary}

\subsection{Getting to the Point Process}

A measure \(\mu\) is called \textbf{locally finite}, if \(\mu(A) < \infty\) for
all bounded \(A\in\borel(\real^\dimension)\). It is called \textbf{simple}, if
\(\mu(\{x\})\in\{0,1\}\).
Let \(\locFiniteMeasure\) be the \textbf{set of all locally finite counting measures}
on \(\real^\dimension\) and \(\locFiniteMeasAlg\) the smallest \(\sigma\)-algebra
such that
\[
	\pi_B : \begin{cases}
		\locFiniteMeasure \to \borel(\real)\\
		\mu \mapsto \mu(B)
	\end{cases}
\]
is measurable for every \(B\in\borel(\real^\dimension)\). I.e.
\[
	\locFiniteMeasAlg := \sigma(\pi_B : B\in\borel(\real^\dimension))
\]

\begin{definition}[Point Process]
	A \textbf{point process} is a measurable mapping \(\Phi:\Omega\to\locFiniteMeasure\)
	from a probability space \((\Omega, \mathcal{A}, \Pr)\) into the set of
	locally finite counting measures. Due to the definition of \(\locFiniteMeasAlg\)
	\[
		\Phi(B) := \pi_B\circ \Phi : \Omega \to \real
	\]
	is a random variable for any \(B\in\borel(\real^\dimension)\). If
	\[
		\Lambda:\begin{cases}
			\borel(\real^\dimension)\to [0,\infty)\\
			B \mapsto \E[\Phi(B)]
		\end{cases}
	\]
	is locally finite, then it is called the \textbf{intensity measure}
	(expectation measure) of \(\Phi\).  If \(\Lambda(dx) = \lambda dx\) for some
	constant \(\lambda\in\real\), then \(\lambda\) is called the
	\textbf{intensity} of \(\Phi\).
	A point process is called \textbf{simple}, if \(\Phi(\omega)\) is simple
	for every \(\omega\in\Omega\).
\end{definition}

\begin{definition}[Marked Point Process]
	A point process \(\Phi\) on \(\locFiniteMeasure[\dimension+n]\) is called
	\textbf{marked point process}, if \(\Phi_p = \Phi(\cdot \times \real^n)\)
	is a point process on \(\locFiniteMeasure\).

	A marked point process is called \textbf{stationary}, if for all
	\(h\in\real^\dimension\), \(A_i\in\borel(\real^\dimension)\),
	\(M_i\in\borel(\real^n)\) and \(B_i\subseteq\nat_0\) we have
	\[
		\Pr(\Phi(A_i\times M_i) \in B_i, i=1,2,\dots)
		= \Pr(\Phi(A_i+h \times M_i)\in B_i, i=1,2,\dots)
	\]
\end{definition}

\begin{lemma}[Campbell]\label{lem: Campbell}
	For a stationary marked point process \(\Phi\) with intensity \(\Lambda\),
	we have	
	\[
		\Lambda(A \times L) = \lambda(L) |A|
	\]
	where \(|A|\) is the lebesgue measure of \(A\). We call
	\[
		\Pr_M(L) = \frac{\lambda(L)}{\lambda(\real^n)}
	\]
	the \textbf{distribution of marks}, because we have for integrable \(f\)
	(Campbell)
	\[
		\E\left[\int f(x,m) \Phi(dx,dm)\right]
		= \lambda \int \int f(x,m) \Pr_M(dm) dx
	\]
\end{lemma}
\begin{proof}
	\fxwarning{TODO: Fakt in 3.5 räumliche statistik skript schlather}
\end{proof}

\subsection{Making it Count}

\begin{definition}[Counting Critical Points]
	Let \(\negEV(H)\) be the number of negative eigenvalues of
	\(H\in\real^{\dimension\times\dimension}\), let \(\rf\) be a random field.
	Then we implicitly\footnote{
		\(\borel(\real^\dimension)\times\borel(\real)\) is a semiring. So by
		Carathéodory's extension theorem there is a unique extension of
		\(\crit^\alpha\) to
		\(\sigma(\borel(\real^\dimension)\times\borel(\real))=\borel(\real^{\dimension+1})\).
		Use this extension to obtain \(\crit^\alpha(\omega)\in\locFiniteMeasure[\dimension+1]\).
	} define the marked point process of critical points as
	\begin{align*}
		\crit^\alpha:
		\begin{cases}
			\borel(\real^\dimension)\times\borel(\real) \to \real\\
			(\Vol, A) \mapsto
			\#\left\{t\in\Vol:
				\nabla\rf(t)=0,
				\frac{\rf(t)}{\dimension}\in A,
				\negEV(\nabla^2 \rf(t))=\dimension\alpha
			\right\}.
		\end{cases}
	\end{align*}
	Here the critical points are the points and the height \(\frac{\rf(t)}\dimension\)
	are the marks. For \(\alpha=0\) we are counting the critical points where all
	eigenvalues are positive. In other words: \(\crit^0\) are the minima,
	\(\crit^1\) are the maxima, all others are saddle points.
\end{definition}
\begin{proof}[Proof (Measurability)]
	First note that	
	\begin{align*}
		\locFiniteMeasAlg[\dimension+1]
		&= \sigma(\pi_A : A\in\borel(\real^{\dimension+1}))\\
		&=\sigma(\underbrace{
			\{\pi_A^{-1}(B) : A\in\borel(\real^{\dimension+1}), B\in\borel(\real)\}
		}_{=:\mathcal{E}}
		)
	\end{align*}
	since it is sufficient to prove measurability on the generator
	\(\mathcal{E}\), we only need that the following sets are measurable
	\begin{align*}
		(\crit^\alpha)^{-1}(\pi^{-1}_{ A }(B))
		&= (\pi_{A} \circ \crit^\alpha)^{-1}(B)
		= (\crit^\alpha(A))^{-1}(B)\\
		&= \#\left\{\left(t,\frac{\rf(t)}{\dimension}\right) \in A:
			\nabla\rf(t)=0,
			\negEV(\nabla^2 \rf(t))=\dimension\alpha
		\right\}\\
		&= \lim_{\epsilon\to 0} \sum_{k\in I^\epsilon}
		\ind\left\{
			\inf_{t\in \left(\text{id}, \frac{\rf}\dimension, \negEV(\nabla^2\rf)\right)^{-1}(A^\epsilon_k\times \{\dimension\alpha\}) }
		|\nabla \rf(t)| = 0
		\right\}
	\end{align*}
	\fxwarning{probably not quite right yet}
	where \((A^\epsilon_k)_{k\in\nat}\) is an \(\epsilon\) tiling of \(A\) (countable).
\end{proof}

\begin{lemma}[Stationarity]
	If \(\rf\) is stationary, \(\crit^\alpha\) is stationary.
\end{lemma}
\begin{proof}
	\fxwarning{TODO}
\end{proof}




\begin{theorem}[Kac-Rice Formula]
	For a centered \fxnote*{needed? careful: if removed then stationary
	(covariance) and strong stationarity (distribution is the same) is not the
	same anymore. Need strong stationarity to replace \(t\) with \(0\). Also conditional
	expectation argument breaks
	}{Gaussian}
	random field \(\rf\) the intensity measure of the critical points
	\(\crit^\alpha\) with index \(\dimension\alpha\) is given by
	\begin{align*}
		&\Lambda^\alpha(\Vol,A)
		=\E[\crit^\alpha(\Vol, A)]\\
		&= \int_\Vol \E\left[
			\left|\det(\nabla^2 \rf(t))\right|
			\ind_{\frac{\rf(t)}{\dimension}\in A} \ind_{\negEV(\nabla^2 \rf(t))=\dimension \alpha}
			\bigm| \nabla\rf(t) = 0 
		\right] \density_{\nabla\rf(t)}(0)dt
		\\
		\overset{\text{stationary}}&=
		|\Vol|\ \underbrace{\E\left[
			|\det(\nabla^2 \rf(0))|
			\ind_{\frac{\rf(0)}\dimension\in A} \ind_{\negEV(\nabla^2 \rf(0))=\dimension \alpha}
		\right] \density_{\nabla\rf(0)}(0)}_{=:\lambda(A)}
	\end{align*}
	where \(\density_{\nabla\rf(t)}\) is the density of \(\nabla\rf(t)\). The
	second equality is only true when \(\rf\) is stationary.
\end{theorem}

\begin{proof}
	The first inequality essentially follows from \textcite[Theorem 6.4]{azaisLevelSetsExtrema2009}
	about the expected number of weighted roots. The weighting are the indicator
	functions here. For completeness we reproduce this proof in the appendix.

	The second inequality follows as \(\nabla\rf(t)\) is independent of
	\((\rf(t),\nabla^2\rf(t))\) due to Corollary~\ref{cor: uncorr leads to indep
	in gaussian case} which turns the conditional expectation into a normal
	expectation. Lastly replace all \(t\) with \(0\) due to stationarity.
\end{proof}

\begin{corollary}
	 With Lemma~\ref{lem: Campbell},
	we have the height distribution of critical points with index \(\alpha\)
	\begin{align*}
		\Pr_{\rf\mid \text{crit}(\alpha)}(A)
		&= \frac{\lambda(A)}{\lambda(\real)}
		= \frac{
			\E[|\det(\nabla^2 \rf(0))|
			\ind_{\negEV(\nabla^2\rf(0))=\dimension\alpha}
			\ind_{\frac{\rf(0)}\dimension \in A}
			]
		}{
			\E[|\det(\nabla^2 \rf(0))|\ind_{\negEV(\nabla^2\rf(0))=\dimension\alpha}]
		}\\
		&= \frac{
			\int \int_A p(H, \dimension \epsilon) |\det(H)|
			\ind_{\negEV(H)=\dimension\alpha}d\epsilon dH
		}{
			\int \int p(H, \dimension \epsilon) |\det(H)|
			\ind_{\negEV(H)=\dimension\alpha}d\epsilon dH
		}
	\end{align*}
	where \(p\) is the joint density of \((\nabla^2 \rf(0), \rf(0))\). Therefore
	the density of \(\Pr_{\rf\mid\text{crit}(\alpha)}\) is up to a normalizing
	constant
	\begin{equation*}
		\Omega^\alpha(\epsilon)
		= \int |\det(H)| \ind_{\negEV(H) = \dimension\alpha} p(H, \dimension\epsilon)dH,
	\end{equation*}
\end{corollary}