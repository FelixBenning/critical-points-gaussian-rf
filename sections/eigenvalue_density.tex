
\section{Eigenvalue Density}

\subsection{Random Matrix Background Theory}

\begin{definition}[Spaces]\label{def: differenSpacesForRMT}
    Let 
    \begin{align*}
        \jacobiSpace & := \real^\dims\times\real^{\dims-1}_{>0} \\
        \weightSpace & := \left\{ (p_1, \ldots, p_{N-1}) \mid p_i > 0 \sum_{i=1}^{N-1}p_i < 1 \right\} \subseteq \real^{\dims-1} & \left( p_N := 1-\sum_{i=1}^{N-1} p_i \right) \\
        \orderedRSpace & := \{ x\in\real^\dims \mid x_1 < x_2 < \ldots < x_\dims \} \\
        \nPointMSpace & := \orderedRSpace \times \weightSpace.
    \end{align*}
    The set $\jacobiSpace$ parametrises two sequences $(a_i)_{i=1,\ldots,\dims}$ in $\real$ and $(b_i)_{i=1,\ldots,\dims-1}$ in $\real_{>0}$. The set $\weightSpace$ parametrises $\dims$ weights that sum up to $1$ and which are greater than zero. $\orderedRSpace$ parametrises an ordered sequence of $\dims$ distinct points and finally $\nPointMSpace$ is the space of measures on exactly $\dims$ points in $\real$.
\end{definition}

\begin{definition}[Jacobi-Matrices]
    We define a Jacobi matrix to be the tridiagonal matrix obtained from choosing a point $(a,b) \in \jacobiSpace$, which looks like
    \begin{align*}
        \jacobiMat :=
        \begin{pmatrix}
            a_1    & b_1 & 0      & 0      & \ldots           & 0                \\
            b_1    & a_2 & b_2    & 0      & \ldots           & 0                \\
            0      & b_2 & a_3    & b_3    & \ldots           & 0                \\
            0      & 0   & \ddots & \ddots & \ddots           & \vdots           \\
            \vdots &     &        & \ddots & \ddots           & b_{\dims-1} \\
            0      &  0  & \ldots & 0      & b_{\dims-1} & a_{\dims}   
        \end{pmatrix}
    \end{align*}
    It should be clear when we refer to the Jacobian (!) matrix as the matrix of partial differentials of functions and when we refer to a Jacobi (!) matrix in the tridiagonal sense.
\end{definition}

Since Jacobi Matrices are symmetric they can be diagonalised via orthonormal matrices. This allows us to define the following.

\begin{definition}[Spectral Measure of Jacobi Matrices]
    If $\jacobiMat = QDQ^T$ with $Q$ orthonormal and $D = \operatorname{diag}(\evGOE_1, \ldots, \evGOE_\dims)$, we define
    \begin{align*}
        \spectralDiffeo(\jacobiMat) := \sum_{i=1}^{\dims} (Q^T e_1)^2_i \delta_{\evGOE_i},
    \end{align*}
    where $e_1 = (1,0,\ldots,0)^T$.
\end{definition}

\begin{theorem}\label{thm: trafoTheorem}
    The map $\spectralDiffeo: \jacobiSpace \to \nPointMSpace$ that assigns a Jacobi-Matrix its spectral measure is actually a diffeomorphism. Furthermore the following points hold:
    \begin{enumerate}[label=(\alph*)]
        \item If $\spectralDiffeo(\jacobiMat) = \sum_{i=1}^\dims p_i \delta_{\evGOE_i}$ then
        \begin{align}
            \prod_{k=1}^{\dims-1} b_k^{2(\dims-k)} = \prod_{k=1}^\dims p_k \cdot \prod_{i<j} |\evGOE_i - \evGOE_j|^2
        \end{align}
        \item The absolute value of the Jacobian determinant of the map $\spectralDiffeo^{-1}$ equals
        \begin{align}\label{inverseSpectralDet}
            \frac{\prod_{k=1}^{\dims-1} b_k}{2^{\dims-1} \prod_{k=1}^\dims p_k} 
            = \frac{\prod_{k=1}^\dims p_k \cdot \prod_{i<j} |\evGOE_i - \evGOE_j|^4}{2^{\dims-1} \prod_{k=1}^{\dims-1} b_k^{4(\dims - k)-1}}.
        \end{align}
    \end{enumerate}
\end{theorem}
\begin{proof}
    \fxnote{Verweis einfügen, viel zu lang der beweis und wäre nur abschreiben}
\end{proof}

The differentiable one-to-one correspondence between Jacobi-Matrices and the space of measures on exactly $\dims$ real points allows us to infer the exact eigenvalue distribution given the distribution of the entries of the Jacobi-Matrix.

\subsection{Orthogonal Transformation}

Let $p(H, \rf)$ be the joint density of a symmetric Gaussian random matrix $H$ (every entry is normal distributed) and a Gaussian $\rf$, where the entries on the diagonal of $H$ and the random varialbe $\rf$ may be correlated. The off-diagonal entries are all independent of each other, of the diagonal of $H$ and of $\rf$.

Furthermore, let $p(H,\rf)$ be such that the dependency on $H$ is reduced to dependency in terms of $\tr(H^k)$, where $k$ is a positive integer. Then $p(OHO^T, \rf) = p(H,\rf)$ for orthogonal $O$ \fxnote{ independent of $H$ and $\rf$ ? }

Our goal is to transform any $H$ into a certain tridiagonal form - for which we know the eigenvalue distribution.

\begin{lemma}[Householder-Matrices]
    Let $v \in \real^\dims$. Then there exists an orthogonal matrix $P$, such that $Pv = (\norm{v}, 0, \ldots, 0)^T$.
\end{lemma}
\begin{proof}
    Let \(v_1:= \frac{v}{\|v\|}\) and find (any) orthonormal basis
    \(v_2,\dots,v_\dims\) of \(v_1^\perp\). Then \(P:=(v_1\dots
    v_\dims)^T\) has the desired properties. 
\end{proof}

\begin{lemma}\label{lem: trafoLemma}
    There exists an orthogonal matrix $O$, such that $OHO^T$ has the form of a Jacobian $\jacobiMat$ with 
    $$(a_1, \ldots, a_\dims, \rf) \eqd (H_{11}, \ldots, H_{\dims \dims}, \rf),$$ which is independent of the independent vector 
    $$(b_1, \ldots, b_{\dims-1}) \eqd \left( \sqrt{\sum_{k=2}^{\dims} H_{k 1}^2}, \ldots, \sqrt{\sum_{k=\dims}^{\dims} H_{k (\dims-1)}^2} \right).$$
\end{lemma}
\begin{proof}
    \fxnote{tadsächlich ausführlicher bidde :3}
\end{proof}

What \ref{lem: trafoLemma} allows us to do is the following change of measures:
\begin{equation}\label{eq: trafoEquation}
    \begin{split}
        \int p(H,\rf) \prod_{i\leq j} \dd h_{ij} \dd \rf
        & = \int p(\jacobiMat, \rf) \prod_{i\leq j} \dd h_{ij} \dd \rf \\
        & = \int \widetilde{p}(\jacobiMat, \rf) 
        \prod_{i=1}^\dims \dd h_{i i} \dd \rf 
        \prod_{i=1}^{\dims-1} \prod_{k=i+1}^{\dims} p_{H_{k i}}(h_{k i}) \dd h_{k i}\\
        & = \int \widetilde{p}(\jacobiMat, \rf) 
        \prod_{i=1}^\dims \dd h_{i i} \dd \rf 
        \prod_{i=1}^{\dims-1} p_{B_i}(b_i) \dd b_i .
    \end{split}
\end{equation}

All we did was use the independence of $(a,\rf)$ from the off-diagonal terms to separate the density that is unchanged when turned into Jacobi-Matrix form. Then we switched the density of the off-diagonal terms with the density of the square root of sums of squares of Gaussians, which we will explain further.\\
In the first step we implicitly used the transformation theorem with the operation $H \to OHO^T$ for $O$ orthogonal, which has a Jacobian (!) equal to one.

The important thing to note is that we now have a density of $(H,\rf)$ in terms of random variables $(a,b,\rf)$. Now to obtain the eigenvalue distribution all we need to do is to apply the transformation theorem using Theorem \ref{thm: trafoTheorem} and integrate out the $p_i$.

We will need a basic fact about gamma distributions to continue.

\begin{lemma}[Facts about the Gamma distribution]\label{lem: gammaFacts}
    We are using the gamma distribution parametrised by $\Gamma(\theta, r)$ with density
    $$ \frac{\theta^r}{\Gamma(r)} x^{r-1} e^{\theta x} \ind_{x>0}$$
    given in $x$.
    \begin{enumerate}
        \item Let $X_1, \ldots, X_\dims$ be i.i.d. $\nn(0,\secCorrFactor)$-distributed random        variables. Then 
            $$ X_1^2 + \ldots + X_\dims^2 \eqd \Gamma\left(\frac{1}{2\secCorrFactor}, \frac{\dims}{2}\right).$$

        \item If $Y \sim \Gamma(\theta, r)$, then $\sqrt{Y}$ has density
            $$ \frac{2\theta^r}{\Gamma(r)} x^{2r-1} e^{-\theta x^2} \ind_{x>0} $$
    \end{enumerate}
\end{lemma}
\begin{proof}
    Inspect densities and use substitution. \fxnote{more detail?}
\end{proof}



\subsection{Application to our problem}

We are given a symmetric matrix $H$ that has centered Gaussians on its diagonal whose correlation is given by 
$$  \E[H_{ii}H_{jj}] = (1+2\delta_{ij}) \secCorrFactor$$
with $\secCorrFactor := \frac{\sqC''(0)}{N}$. Their correlation to $\rf$ is given By
$$ \E[H_{ii} \rf] = \firstCorrFactor $$
with $\firstCorrFactor := \sqC'(0)$.

The upper diagonal is given by i.i.d. centered Gaussians with variance $\secCorrFactor$, which are also independent of the diagonal and $\rf$.

If we want to use Equation \ref{eq: trafoEquation} we will need to figure out the factor
$$ \frac{\prod_{i=1}^{\dims-1} p_{B_i}(b_i)}{\prod_{i=1}^{\dims-1}\prod_{k=i+1}^{\dims} p_{H_{k i}}(h_{k i})} $$

We can immediately write down the denominator as 
\begin{align*}
    \prod_{i=1}^{\dims-1}\prod_{k=i+1}^{\dims} \frac{1}{\sqrt{2\pi\secCorrFactor}} e^{-\frac{h_{ki}^2}{2\secCorrFactor}}
    & = \left( \frac{1}{\sqrt{2\pi\secCorrFactor}} \right)^{\frac{\dims(\dims-1)}{2}} e^{-\frac{1}{2\secCorrFactor} \sum_{i=1}^{\dims-1}\sum_{k=i+1}^{\dims} h_{ki}^2} \\
    & = \left( 2\pi\secCorrFactor \right)^{-\frac{\dims(\dims-1)}{4}} e^{-\frac{1}{4\secCorrFactor} \left( \tr(H^2) - \sum_{i=1}^\dims h_{ii} \right)}
\end{align*}

Using Lemma \ref{lem: gammaFacts} we infer that the numerator is equal to
\begin{align*}
    & \prod_{i=1}^{\dims-1} \frac{2 b_i^{\dims-i-1}}{(2\secCorrFactor)^{\frac{\dims-i}{2}}\Gamma(\frac{\dims-i}{2})}  e^{-\frac{1}{2\secCorrFactor} b_i^2} \ind_{b_i>0} \\
    & = \frac{2^{N-1}}{(2\secCorrFactor)^{\frac{\dims(\dims-1)}{2}} \prod_{i=1}^{\dims-1} \Gamma\left(\frac{i}{2}\right)} \left(\prod_{i=1}^{\dims-1} b_i^{\dims-i-1} \ind_{b_i>0} \right)
    e^{-\frac{1}{2\secCorrFactor}\sum_{i=1}^{\dims-1} b_i^2}
\end{align*}

Given the density $p(H,\rf)$ from Lemma \ref{lem: joint density} we define 
\begin{align*}
    \widetilde{p}(H,\rf):= C_\dims' \exp\left( -\dims^2 \widetilde{S}_2'(H) - \dims S_1(H,\rf) \right)
\end{align*}
with
\begin{align*}
    \dims^2\widetilde{S}_2'(H) := \frac{1}{4 \secCorrFactor} \left[ \sum_{i=1}^\dims h_{ii}^2 - \frac{1}{2+\dims} (\tr(H))^2 \right].
\end{align*}

If we plug all this into Equation \ref{eq: trafoEquation} we obtain
\begin{align*}
    & \int p(H,\rf) \prod_{i\leq j} \dd h_{ij} \dd \rf \\
    & = \int \widetilde{p}(\jacobiMat,\rf) \left(\prod_{i=1}^{\dims-1} b_i^{\dims-i-1} \ind_{b_i>0} \right) e^{-\frac{1}{2\secCorrFactor}\sum_{i=1}^{\dims-1} b_i^2} \left(\prod_{i=1}^{\dims} \dd a_{i}\right) \left(\prod_{i=1}^{\dims-1} \dd b_{i}\right) \dd \rf
    \\
    & = \int \operatorname{const} \cdot p(\jacobiMat,\rf) \left(\prod_{i=1}^{\dims-1} b_i^{\dims-i-1} \ind_{b_i>0} \right) \left(\prod_{i=1}^{\dims} \dd a_{i}\right) \left(\prod_{i=1}^{\dims-1} \dd b_{i}\right) \dd \rf
\end{align*}

We now want to transform this Jacobi-Matrix density into an eigenvalue density with the help of Theorem \ref{thm: trafoTheorem}. Since we know that
$$ \tr(\jacobiMat) = \sum_{i=1}^\dims \evGOE_i \qquad \text{and} \qquad \tr((\jacobiMat)^2) = \sum_{i=1}^\dims \evGOE_i^2 $$
we can ignore the density $p(\jacobiMat,\rf)$ for now and take a look at what happens to the $b_i$-factor multiplied by the Jacobian (!) of the density of the diffeomorphism $\spectralDiffeo$:
\begin{align*}
    & \left(\prod_{k=1}^{\dims-1} b_k^{\dims-k-1} \ind_{b_k>0} \right)
    \frac{\prod_{k=1}^\dims p_k \cdot \prod_{i<j} |\evGOE_i - \evGOE_j|^4}{2^{\dims-1} \prod_{k=1}^{\dims-1} b_k^{4(\dims - k)-1}} \\
    & = \frac{1}{2^{\dims-1}}  \frac{\prod_{k=1}^\dims p_k \cdot \prod_{i<j} |\evGOE_i - \evGOE_j|^4}{\prod_{k=1}^{\dims-1} b_k^{3(\dims - k)}} \\
    & = \frac{1}{2^{\dims-1}} \prod_{i<j} |\evGOE_i - \evGOE_j| \prod_{k=1}^\dims p_k^{-\frac{1}{2}}.
\end{align*}

The last equality made use of Theorem \ref{thm: trafoTheorem} (a). To integrate out the $p_k$ we use the following well known definition.

\begin{definition}[Dirichlet distribution]
    The distribution on $\real^{N-1}$ with density
    \begin{align*}
        \frac{1}{D_N} p_1^{\beta_1 - 1} \cdot \ldots \cdot p_\dims^{\beta_\dims-1} \ind_{\weightSpace}
    \end{align*}
    is called \textit{Dirichlet distribution} with parameters $\dims$ and $\beta_1,\ldots,\beta_\dims > 0$, where
    \begin{align*}
        D_N := \frac{\Gamma(\beta_1)\cdot \ldots \cdot \Gamma(\beta_{\dims})}{\Gamma(\beta_1+\ldots+\beta_\dims)}
    \end{align*}
    and $\weightSpace$ is the space of weights of an $\dims$-point measure introduced in Definition \ref{def: differenSpacesForRMT}.
\end{definition}

Thus, if we write $\evGOE = (\evGOE_1, \ldots, \evGOE_\dims)$ the joint eigenvalue distribution together with $\rf$ will be equal to 
\begin{align*}
    \rho(\evGOE,\rf) = \operatorname{const} \cdot \exp(- \dims^2 \mathcal{S}_2(\evGOE) - \dims \mathcal{S}_1 (\evGOE, \rf)) \prod_{i<j} |\evGOE_i - \evGOE_j| \ind_{\evGOE_1 < \ldots < \evGOE_\dims},
\end{align*}
where
\begin{align*}
    \dims^2 \mathcal{S}_2(\evGOE) = \frac{1}{4\secCorrFactor}  \left[ \sum_{i=1}^\dims \evGOE_i^2 - \frac{1}{2+\dims} \left(\sum_{i=1}^\dims \evGOE_i \right)^2 \right]
\end{align*}
and
\begin{align*}
    \mathcal{S}_1(\evGOE, \rf) = \frac1{2\sqC''(0)Q}\left[
        \tfrac{\dims+2}{\dims^3}\rf^2
        - \tfrac{2\sqC'(0)}{\dims^2\sqC''(0)}\rf\left(\sum_{i=1}^\dims \evGOE_i \right)
        + \tfrac{\sqC'(0)^2}{\sqC''(0)^2\dims(\dims+2)}\left(\sum_{i=1}^\dims \evGOE_i \right)^2
    \right],
\end{align*}
and
\[
	Q = \left(1+\frac2\dims\right)
	\frac{\sqC(0)}{\sqC''(0)} - \frac{\sqC'(0)^2}{\sqC''(0)^2}.
\]

The constant is dependent on $N$ and it is equal to \fxnote{Konstante ausrechnen :(}.