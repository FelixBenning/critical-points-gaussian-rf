\section{Joint Density Derivation}\label{sec: join density derivation}

In this section we want to derive the joint density of \((\nabla^2\rf,\rf):=
(\nabla^2\rf(0), \rf(0))\). As \(\partial_{ij}\rf\) are limits of differences
of multivariate normal random variables, this is a centered multivariate normal
random vector. We therefore only need to consider the covariance. Additionally
we know \(\partial_{ij}\rf = \partial_{ji}\rf\), so we are only interested
in \(((\partial_{ij}\rf)_{i\le j}, \rf)\).

As for the covariances, we have \fxnote{reference}
\begin{align*}
	\Cov(\phi,\phi) &= \dimension \sqC(0)\\
	\Cov(\partial_{ij}\phi,\phi) &= \delta_{ij} \sqC'(0)\\
	\Cov(\partial_{ij}\phi, \partial_{kl}\phi) &= \frac{\sqC''(0)}\dimension
	[\delta_{ij}\delta_{kl} + \delta_{ik}\delta_{jl} + \delta_{il}\delta_{kj}].
\end{align*}
In particular for \(i<j\), we have
\[
	\Cov(\partial_{ij}\phi, \phi) = 0,
\]
and also
\[
	\Cov(\partial_{ij}\phi, \partial_{kl}\phi) = \frac{\sqC''(0)}\dimension
	[\delta_{ik}\delta_{jl} + \underbrace{\delta_{il}\delta_{kj}}_{=0}],
\]
because \(\delta_{il}\delta_{kj}=1\) would imply \(i < j = k \le l = i\), which
is a contradiction. So this covariance is only non-zero, if \(i=k\) and \(j=l\),
i.e. \(\partial_{ij}\phi\) for \(i<j\) has covariance zero with all other terms.
As we are in the multivariate Gaussian case, this also implies independence.
Shifting all these independent mixed derivatives to the end, our covariance
matrix looks like
\[
	\Sigma := \Cov((\partial_{ii}\phi)_{i\le\dimension}, \phi, (\partial_{ij}\phi)_{i<j})
	= \begin{pmatrix}
		S & 0\\
		0 & \frac{\sqC''(0)}{\dimension}\identity
	\end{pmatrix}
\]
with
\[
	S = \begin{pmatrix}
		3\frac{\sqC''(0)}\dimension &  \cdots & \frac{\sqC''(0)}\dimension
		& \sqC'(0) \\
		\vdots & \ddots &\vdots & \vdots \\
		\frac{\sqC''(0)}\dimension & \cdots &  3 \frac{\sqC''(0)}\dimension & \sqC'(0)\\
		\sqC'(0)	& \cdots & \sqC'(0) & \dimension \sqC(0)
	\end{pmatrix}
	=: \begin{pmatrix}
		P & b \\
		b^T & d
	\end{pmatrix}.
\]
Inverting \(\Sigma\) is easy, if we can invert \(S\). If we can invert the
covariance of partial derivatives \(P\), we have
\[
	S^{-1} = \frac1{d- b^T P^{-1}b}\begin{pmatrix}
		( d- b^T P^{-1}b )P^{-1} + P^{-1}b b^T P^{-1} & -P^{-1} b\\
		-b^TP^{-1} &  1
	\end{pmatrix},
\]
as one can verify by calculating \(SS^{-1}\). As \(P\) is of the form
\[
	P=\frac{\sqC''(0)}{\dimension}[2\identity + \ones\ones^T]
\]
with \(\ones=(1,\dots,1)^T\in\real^\dimension\), we can use the Sherman-Morrison
formula.

\begin{lemma}[Sherman-Morrison Formula]
	Let \(A\) be an invertible matrix, \(u,v\in\real^\dimension\) column vectors.
	Then \(A+ uv^T\) is invertible iff \(1+v^T A^{-1}u \neq 0\), and if that is
	the case, we have
	\[
		(A+uv^T) = A^{-1} + \frac{A^{-1}u v^T A^{-1}}{1+v^T A^{-1}u}.
	\]
\end{lemma}
\begin{proof}
	\fxnote{find reference (or reproduce wikipedia proof)}
\end{proof}

The application of this formula results in
\[
	P^{-1}
	= \frac{\dimension}{\sqC''(0)}\left[
		\tfrac12\identity + \frac{\tfrac14\ones\ones^T}{1+\tfrac12\ones^T\ones}
	\right]
	= \frac{\dimension}{2\sqC''(0)}\left[
		\identity + \frac{1}{2+\dimension}\ones\ones^T
	\right].
\]
