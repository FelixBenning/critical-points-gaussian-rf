\section{Kac-Rice}

\begin{definition}[Random Field]
	A collection of random variables \((\rf(t))_{t\in\real^\dimension}\) is called
	\emph{random field} over \(\real^\dimension\). The \emph{covariance function}
	is defined as
	\begin{equation*}
		\C(x,y) = \Cov(\rf(x), \rf(y)).
	\end{equation*}
	A random field is called \emph{stationary}, if
	\begin{equation*}
		\C(x,y) = \C(x-y).
	\end{equation*}
	It is called \emph{isotropic} (rotation invariant) if	
	\begin{equation*}
		\C(x,y) = \dimension \sqC\left(\frac{\|x-y\|^2}{2\dimension}\right).
	\end{equation*}
\end{definition}

\begin{definition}[Counting Critical Points]
	Let \(\negEV(H)\) be the number of negative eigenvalues of
	\(H\). Then we define for some volume \(\Vol\in\borel(\real^\dimension)\),
	\(A\in\borel(\real)\), \(\alpha, u\in\real\)
	\begin{align*}
		\crit_u(\alpha, A)
		&:= \crit_u(\Vol, \alpha, A)\\
		&:= \#\left\{t\in\Vol:
			\frac{\rf(t)}{\dimension}\in A,
			\negEV(\nabla^2 \rf(t))=\dimension\alpha,
			\nabla\rf(t)=u
		\right\}	
	\end{align*}
\end{definition}

\begin{theorem}[Kac-Rice Formula]
	For a \fxnote*{needed?}{Gaussian} random field \(\rf\) we have
	\begin{align*}
		&\E[\crit_u(\alpha, A)]\\
		&= \int_\Vol \E\left[
			\left|\det(\nabla^2 \rf(t))\right|
			\ind_{\frac{\rf(t)}{\dimension}\in A} \ind_{\negEV(\nabla^2 \rf(t))=\dimension \alpha}
			\bigm| \nabla\rf(t) = u 
		\right] \density_{\nabla\rf(t)}(u)dt
		\\
		\overset{\text{stationary}}&=
		|\Vol|\ \E\left[
			|\det(\nabla^2 \rf(0))|
			\ind_{\frac{\rf(0)}\dimension\in A} \ind_{\negEV(\nabla^2 \rf(0))=\dimension \alpha}
		\right] \density_{\nabla\rf(0)}(u)
	\end{align*}
	where \(\density_{\nabla\rf(t)}\) is the density of \(\nabla\rf(t)\).
\end{theorem}

\begin{remark}
	Therefore \(\mu_\alpha: A\mapsto \E[\crit_0(\alpha, A)]\) is a measure on
	\(\real\) with probability density
	\begin{equation*}
		|\Vol| \phi_{\nabla\rf(0)}(0) \Omega(\alpha, \epsilon),
	\end{equation*}
	where
	\begin{equation*}
		\Omega(\alpha, \epsilon)
		= \int |\det(H)| \ind_{\negEV(H) = \dimension\alpha} p(H, \dimension\epsilon)dH,
	\end{equation*}
	and \(p(H, \phi)\) is the joint probability density of \((\nabla^2\rf(0), \rf(0))\).

	Notice that \(\mu_0(\real)\) are all critical points where the hessian is
	positive definite, i.e. the number of minimas in \(\Vol\). If we draw a
	minima \(M\) uniformly form the set of minima, then \(\mu_0/\mu_0(\real)\) is their
	scaled height distribution. I.e.
	\[
		\frac{Z(M)}{\dimension} \sim \frac{\mu_0}{\mu_0(\real)}.
	\]
	\fxwarning*{}{This is wrong: We don't draw from the set of minima numbered \(\crit_0(0, \real)\),
	we draw from the expected number of minima \(\E[\crit_0(0,\real)]\).
	This might be fixable if we expand \(\Vol = \Vol_1 + \dots + \Vol_n\) and get
	some sort of law of large numbers 
	\[
		\frac{\crit_0(\Vol, 0, A)}{\crit_0(\Vol, 0, \real)}
		= \frac{
			\tfrac1n \sum_{k=1}^n \crit_0( \Vol_k, 0, A )
		}{
			\tfrac1n \sum_{k=1}^n \crit_0( \Vol_k, 0, \real )
		} \approx \frac{
			\E[\crit_0(\Vol_1, 0, A)]
		}{
			\E[\crit_0(\Vol_1, 0, \real)]
		} = \frac{\mu_0(A)}{\mu_0(\real)}
	\]
	as we get more and more critical points. Unfortunately the number of critical
	points in \(V_k\) and \(V_i\) are not independent. But they are almost uncorrelated
	if they are far apart/large given an appropriate covariance function for \(\rf\).
	}
\end{remark}