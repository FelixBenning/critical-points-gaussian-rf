\section{Kac-Rice}

We define the \textbf{Level Set Count} as
\[
	\level(f, \Vol) := \#\{t\in \Vol : f(t) = u\}.
\]

\begin{prop}[Area formula {\cite[Prop. 6.1]{azaisLevelSetsExtrema2009}}]
	Let \(f\in\contin^1(U, \real^\dimension)\) where \(U\) is an open
	subset of \(\real^\dimension\). Assume that the set of critical values of
	\(f\) has zero Lebesgue measure. Let \(g: \real^\dimension\to\real\) be
	continuous and bounded. Then for any \(\Vol\in\borel(U)\)
	\[
		\int_{\real^\dimension}g(u) \level(f, \Vol)du
		= \int_\Vol |\det(f'(t))| g(f(t))dt
	\]
	Further, for \(h(t,u) = \ind_{t\in T} g(u)\) for \(T\in\borel(U)\), we have
	\[
		\int_{\real^\dimension} \sum_{t\in f^{-1}(u)} h(t, u)du
		= \int_{\real^\dimension} |\det(f'(t))| h(t, f(t))dt.
	\]
\end{prop}
\begin{proof}
	\textcite[pp. 120]{azaisLevelSetsExtrema2009}
\end{proof}

Notice how
\begin{equation}\label{eq: level set and crit point relation}
	\crit^\alpha(\Vol, A)
	= \level[0](
		\nabla\rf,
		\Vol
		\cap \rf^{-1}(\dimension A)
		\cap \negEV(\nabla^2\rf)^{-1}(\{\dimension\alpha\})
	).
\end{equation}
While we do not get a formula for \(\level(f,\Vol)\) for all \(u\) from the area
formula, the integral equation holds for all continuous \(g\), so we get a
formula for almost all \(u\). But we are interested in a specific \(u\), \(u=0\),
so ``\emph{almost} all'' is useless.

This problem will be solved by taking the expectation. It turns out that
the expectation of \(\level\) is actually continuous in \(u\), which then allows
us to generalize ``almost all \(u\)'' to ``all \(u\)''. In particular we then
have a formula for \(u=0\), which interests us. But this continuity argument is,
why we need to do this for general \(u\in\real^\dimension\) even though we are
only interested in \(u=0\).

Now we need \(\nabla \phi \in \contin^1\), this leads to Assumption
\ref{assumption: continuity}.  Additionally we need the set of critical values
of \(f=\nabla\phi\) to have
Lebesgue measure zero, which leads to Assumption \ref{assumption: critical
values of grad phi}


\begin{assumption}[Requirements for Rice Formula]\label{assumptions: rice}
	Let \(\rf: U\to\real^\dimension\) be a random field, assume
	\begin{enumerate}[label=(\roman*)]
		\item\label{assumption: continuity} \(\rf\in\contin^2\) almost surely
		\item\label{assumption: critical values of grad phi}
		\(\Pr(\exists t\in U: \nabla\rf(t) = u, \det(\nabla^2 \phi(t))=0)=0\)
	\end{enumerate}
\end{assumption}

\begin{theorem}[Rice formula in expectation]\label{thm: rice formula in expectation}
	With Assumptions~\ref{assumptions: rice} we get
	\begin{align*}
		&\E[\level(
			\nabla\rf,
			\Vol
			\cap \rf^{-1}(\dimension A)
			\cap \negEV(\nabla^2\rf)^{-1}(\{\dimension\alpha\})
			)
		]\\
		&= \int_\Vol \E\left[
			|\det(\nabla^2 \rf(t))| \ind_{\rf \in \dimension A}
			\ind_{\negEV(\nabla^2\rf\in\dimension\alpha)}
			\bigm| \nabla \rf(t) = u
		\right] p_{\rf(t)}(u)dt
	\end{align*}
	where \(p_{\rf(t)}\) is the \emph{density} of \(\rf(t)\). Compare with
	\eqref{eq: level set and crit point relation} for the interesting special
	case \(u=0\).
\end{theorem}
\begin{proof}(based on \cite[Thm. 6.4]{azaisLevelSetsExtrema2009})

\end{proof}