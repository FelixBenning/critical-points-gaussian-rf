\section{Making it Count}

\begin{definition}[Counting Critical Points]
	Let \(\rf\) be a random field. We implicitly\footnote{
		\(\borel(\real^\dimension)\times\borel(\real^{1+\dimension^2})\) is a semiring. So by
		Carathéodory's extension theorem there is a unique extension of
		\(\pp\) to
		\(\sigma(\borel(\real^\dimension)\times\borel(\real^{1+\dimension^2}))=\borel(\real^{\dimension+1+\dimension^2})\).
		Use this extension to obtain \(\pp(\omega)\in\locFiniteMeasure[\dimension+1+\dimension^2]\).
	} define the marked point process of critical points as
	\begin{align*}
		\pp:
		\begin{cases}
			\borel(\real^\dimension)\times\borel(\real^{1+\dimension^2}) \to \real\\
			(\Vol, M) \mapsto
			\#\left\{t\in\Vol:
				\nabla\rf(t)=0,
				(\rf(t), \nabla^2 \rf(t))\in M
			\right\}.
		\end{cases}
	\end{align*}
	This is a marked point process with \((\rf(t), \nabla^2\rf(t))\) as marks.
	Let the index of \(H\), be the share of negative eigenvalues\footnote{
		Since the eigenvalues are continuous in the entries of the matrix, the
		number of negative eigenvalues is a measurable function in \(H\).
	}
	\[
		\negEV(H) := \negEV(\evH) := \frac1\dimension \sum_{k=1}^\dimension \ind_{\evH_i < 0},
	\]
	where \(\evH=(\evH_1,\dots,\evH_\dimension)\) are the eigenvalues of
	\(H\in\real^{\dimension\times\dimension}\).
	The point-process of critical points with index \(\alpha\) is therefore
	\[
		\pp^\alpha(\Vol, A)
		:= \pp(\Vol, A\times\negEV^{-1}(\dimension\alpha)).
	\]
	In particular for \(\alpha=0\) we have the point process of minimas\footnote{
		Technically ``no negative eigenvalues'' is only a necessary criterion for a
		minima. But as eigenvalues will almost surely never be zero, we can
		ignore this issue. If one wanted to convince themselves that this is
		no issue, they could use ``\(\le\)'' in the definition of \(\negEV\) to
		obtain \(\tilde{\negEV}\). For the height distribution of minima one
		could then use a sandwich argument on
		\[
			\Pr(\rf_T\in A, \negEV(H_T) = 0)
			\ge \Pr(\rf_T\in A, T\text{ is min})
			\ge \Pr(\rf_T\in A, \tilde{\negEV}(H_T) = 0),
		\]
		with notation from Cor.~\ref{cor: mark distribution}.
		By doing the same steps with \(\tilde{\negEV}\) as with
		\(\negEV\), one can convince themselves about equality of the first and
		last term. This then directly translates to the conditionals used in
		Sec.~\ref{sec: height distribution}.
	}\(\pp^0\).
\end{definition}
\begin{proof}[Proof (Measurability)]
	First note that	
	\begin{align*}
		\locFiniteMeasAlg[\dimension+1+\dimension^2]
		&= \sigma(\pi_B : B\in\borel(\real^{\dimension+1+\dimension^2}))\\
		&=\sigma(\underbrace{
			\{\pi_B^{-1}(C) : B\in\borel(\real^{\dimension+1+\dimension^2}), C\in\borel(\real)\}
		}_{=:\mathcal{E}}
		)
	\end{align*}
	since it is sufficient to prove measurability on the generator
	\(\mathcal{E}\), we only need that the following sets are measurable
	\begin{align*}
		(\pp)^{-1}(\pi^{-1}_{B}(C))
		&= (\pi_{B} \circ \pp)^{-1}(C)
		= (\pp(B))^{-1}(C)\\
		&= \#\left\{\left(t,\rf(t),\nabla^2\rf(t)\right) \in B:
			\nabla\rf(t)=0
		\right\}\\
		&= \lim_{\epsilon\to 0} \sum_{k\in I^\epsilon}
		\ind\left\{
			\inf_{t\in \left(\text{id}, \rf, \nabla^2\rf\right)^{-1}(B^\epsilon_k) }
		|\nabla \rf(t)| = 0
		\right\}
	\end{align*}
	\fxwarning{probably not quite right yet}
	where \((B^\epsilon_k)_{k\in\nat}\) is an \(\epsilon\) tiling of \(B\) (countable).
	We can swap the limit with the series due to monotone convergence.
\end{proof}

\begin{lemma}[Stationarity]
	If \(\rf\) is strictly stationary, \(\pp\) is stationary.
\end{lemma}
\begin{proof}
	Let \(h\in \real^\dimension\) and note that
	\[
		\tilde{\rf}(t) := \rf(t+h)
	\]
	has the same distribution as \(\rf\) since \(\rf\) is strictly stationary. Define
	\(\tilde{\pp}\) using \(\tilde{\rf}\) in place of \(\rf\). Then
	\(\tilde{\pp}\) has the same distribution as \(\pp\).
	And we have
	\begin{align*}
		&\pp(\Vol+h, M)\\
		&= \#\left\{t\in\Vol+h:
			\nabla\rf(t)=0,
			(\rf(t),\nabla^2\rf(t))\in M
		\right\}\\
		&= \#\left\{t\in\Vol:
			\nabla\tilde{\rf}(t)=0,
			(\tilde{\rf}(t),\nabla^2\tilde{\rf}(t))\in M
		\right\}\\
		&= \tilde{\pp}(\Vol, M).
	\end{align*}
	So \(\pp\) is stationary:
	\begin{align*}
		\Pr\left(\pp((\Vol_i+h)\times M_i)\in B_i, i=1,2,\dots\right)
		&= \Pr\left(\tilde{\pp}(\Vol_i\times M_i)\in B_i, i=1,2,\dots\right)\\
		&= \Pr\left(\pp(\Vol_i\times M_i)\in B_i, i=1,2,\dots\right).
		\qedhere
	\end{align*}
\end{proof}


\begin{theorem}[Kac-Rice Formula]
	\label{thm: kac-rice formula}
	For a centered Gaussian random field \(\rf\) satisfying our smoothness
	Assumption~\ref{assmpt: smoothness assumption}, the intensity measure of the
	critical points \(\pp\) is given by
	\begin{align*}
		&\intensityMeas(\Vol,M)
		=\E[\pp(\Vol, M)]\\
		&= \int_\Vol \E\left[
			\left|\det(\nabla^2 \rf(t))\right|
			\ind_M(\rf(t), \nabla^2\rf(t))
			\bigm| \nabla\rf(t) = 0 
		\right] \density_{\nabla\rf(t)}(0)dt
		\\
		\overset{\text{stationary}}&=
		|\Vol|\ \underbrace{\E\left[
			|\det(\nabla^2 \rf(0))|
			\ind_M(\rf(0), \nabla^2\rf(0))
		\right] \density_{\nabla\rf(0)}(0)}_{=:\intensity(M)}
	\end{align*}
	where \(\density_{\nabla\rf(t)}\) is the density of \(\nabla\rf(t)\). The
	second equality is only true when \(\rf\) is stationary.
\end{theorem}

\begin{proof}
	This is an immediate result from Corollary~\ref{cor: gaussian kac-rice}
	\parencite[Corollary~11.2.2]{adlerRandomFieldsGeometry2007}. More precisely
	we define
	\begin{align*}
		f(t) &:= \nabla \rf(t)\\
		g(t) &:= (\rf(t), \nabla^2 \rf(t))
	\end{align*}
	We now simply need to check that the covariance function of the components of
	\(\partial_j f^i(t) = \partial_{ji} \rf(t)\) and \(g^i\) (which are
	also just the second derivatives and \(\rf\) itself) satisfy the smoothness
	requirement of the corollary. As smoothness of the second derivative implies
	smoothness of \(\rf\), we only need to check the requirement for \(\partial_{ji} \rf\).
	In other words, we need
	\[
		\max_{i,j}|\partial_{x_i x_j, y_i y_j}\C(t,t)
		+ \partial_{x_i x_j, y_i y_j}\C(s,s)
		-2\partial_{x_i x_j, y_i y_j}\C(s,t)| \le K |\ln(|t-s|)|^{-(1+\alpha)}
	\]
	But that is \eqref{eq: sufficient for continuous second derivative} in
	Assumption~\ref{assmpt: smoothness assumption} written in terms of
	covariances. By application
	of Corollary~\ref{cor: gaussian kac-rice}, we therefore get
	\begin{align*}
		\intensityMeas(\Vol, M) 
		&= \E[\pp(\Vol, M)]
		= \E[\level(f, g: \Vol, M)]\\
		&= \int_\Vol \E\left[
			|\det \nabla f(t)| \ind_M(\rf(t), \nabla^2\rf(t)) \mid f(t) = 0
		\right] \density_{\nabla\rf(t)}(0)dt.
	\end{align*}
	
	
	The second inequality follows as \(\nabla\rf(t)\) is independent of
	\((\rf(t),\nabla^2\rf(t))\) due to Corollary~\ref{cor: uncorr leads to indep
	in gaussian case} which turns the conditional expectation into a normal
	expectation. Lastly replace all \(t\) with \(0\) due to stationarity.
\end{proof}
\begin{remark}
	As can be seen in Theorem~\ref{thm: general kac-rice} \parencite[Theorem
	11.2.1]{adlerRandomFieldsGeometry2007}, \(\rf\) does not necessarily
	need to be Gaussian. We just need it and its derivatives to satisfy a
	list of properties which are satisfied for Gaussian process with
	smooth enough covariance functions.
\end{remark}

Due to stationarity, we have completely dropped the dependence of \(\rf\) on
\(t\) in Theorem~\ref{thm: kac-rice formula} to define the intensity
\(\intensity\). To reduce notational clutter, we will from now on write
\[
	(\rf,H):= (Z(0), \nabla^2\rf(0))
\]

\begin{corollary}[Mark Distribution of Critical Points in a Stationary Random Field]
	\label{cor: mark distribution}
	Let \(p(z,h)\) be the joint density of \((\rf, H)\) and define
	\[
		\Omega(z,h) := c |det(h)| p(z,h)
	\]
	with \(c\in\real\) such that \(\int \Omega(z,h)dz dh = 1\).
	If we take a random critical point \((T,\rf_T,H_T)\sim \pp\),
	then the distribution over its marks is given by
	\[
		\Pr_{\rf_T,H_T}(A) = \int_A \Omega(z,h) dz dh.
	\]
\end{corollary}
\begin{remark}
	Notice how this distribution of marks is different from
	\[
		\Pr((\rf(t),\nabla^2\rf(t))\in A \mid \nabla\rf(t) = 0)
		\overset{\text{stationary}}
		= \Pr((\rf,H)\in A)
		= \int_A p(z,h)dz dh.
	\]
	In particular the density of marks is \(\Omega\) not \(p\), like it is if
	we simply condition on the fact that \(t\) is a critical point.
\end{remark}
\begin{proof}
	With Lemma~\ref{lem: Campbell}, we have that the distribution of marks is
	given by
	\begin{align*}
		\Pr_{\rf_T,H_T}(L)
		&= \frac{\intensity(L)}{\intensity(\real)}
		\overset{\ref{thm: kac-rice formula}}= \frac{
			\E[|\det(\nabla^2 \rf(0))|
			\ind_L(\rf(0), \nabla^2\rf(0))
			]
		}{
			\E[|\det(\nabla^2 \rf(0))|]
		}\\
		&= \int_L \Omega(z,h) dz dh.
		\qedhere
	\end{align*}
\end{proof}
