\section{Making it Count}

\begin{definition}[Counting Critical Points]
	Let \(\rf\) be a random field. We implicitly\footnote{
		\(\borel(\real^\dims)\times\borel(\real^{1+\dims^2})\) is a semiring. So by
		Carathéodory's extension theorem there is a unique extension of
		\(\pp\) to
		\(\sigma(\borel(\real^\dims)\times\borel(\real^{1+\dims^2}))=\borel(\real^{\dims+1+\dims^2})\).
		Use this extension to obtain \(\pp(\omega)\in\locFiniteMeasure[\dims+1+\dims^2]\).
	} define the marked point process of critical points as
	\begin{align*}
		\pp:
		\begin{cases}
			\borel(\real^\dims)\times\borel(\real^{1+\dims^2}) \to \real\\
			(\Vol, L) \mapsto
			\#\left\{t\in\Vol:
				\nabla\rf(t)=0,
				(\rf(t), \nabla^2 \rf(t))\in L
			\right\}.
		\end{cases}
	\end{align*}
	This is a marked point process with \((\rf(t), \nabla^2\rf(t))\) as marks\footnote{
		Measurability: First note that	
		\begin{align*}
			\locFiniteMeasAlg[\dims+1+\dims^2]
			&= \sigma(\pi_B : B\in\borel(\real^{\dims+1+\dims^2}))\\
			&=\sigma(\underbrace{
				\{\pi_B^{-1}(C) : B\in\borel(\real^{\dims+1+\dims^2}), C\in\borel(\real)\}
			}_{=:\mathcal{E}}.
			)
		\end{align*}
		Since it is sufficient to prove measurability on the generator
		\(\mathcal{E}\), we only need that the following sets are measurable:
		\begin{align*}
			(\pp)^{-1}(\pi^{-1}_{B}(C))
			&= (\pi_{B} \circ \pp)^{-1}(C)
			= (\pp(B))^{-1}(C)\\
			&= \#\left\{\left(t,\rf(t),\nabla^2\rf(t)\right) \in B:
				\nabla\rf(t)=0
			\right\}\\
			&= \lim_{\epsilon\to 0} \sum_{k\in I^\epsilon}
			\ind\left\{
				\inf_{t\in \left(\text{id}, \rf, \nabla^2\rf\right)^{-1}(B^\epsilon_k) }
			|\nabla \rf(t)| = 0
			\right\},
		\end{align*}
		where \((B^\epsilon_k)_{k\in\nat}\) is an \(\epsilon\) tiling of \(B\) (countable).
	}.\fxwarning{Measurability probably not quite right yet}
\end{definition}

\begin{lemma}[Stationarity]
	If \(\rf\) is strictly stationary, \(\pp\) is stationary.
\end{lemma}
\begin{proof}
	Let \(h\in \real^\dims\) and note that
	\[
		\tilde{\rf}(t) := \rf(t+h)
	\]
	has the same distribution as \(\rf\) since \(\rf\) is strictly stationary. Define
	\(\tilde{\pp}\) using \(\tilde{\rf}\) in place of \(\rf\). Then
	\(\tilde{\pp}\) has the same distribution as \(\pp\).
	And we have
	\begin{align*}
		\pp(\Vol+h, L)
		&= \#\left\{t\in\Vol+h:
			\nabla\rf(t)=0,
			(\rf(t),\nabla^2\rf(t))\in L
		\right\}\\
		&= \#\left\{t\in\Vol:
			\nabla\tilde{\rf}(t)=0,
			(\tilde{\rf}(t),\nabla^2\tilde{\rf}(t))\in L
		\right\}
		= \tilde{\pp}(\Vol, L).
	\end{align*}
	So \(\pp\) is stationary:
	\begin{align*}
		\Pr\left(\pp((\Vol_i+h)\times L_i)\in B_i, i=1,2,\dots\right)
		&= \Pr\left(\tilde{\pp}(\Vol_i\times L_i)\in B_i, i=1,2,\dots\right)\\
		&= \Pr\left(\pp(\Vol_i\times L_i)\in B_i, i=1,2,\dots\right).
		\qedhere
	\end{align*}
\end{proof}


\begin{theorem}[Kac-Rice Formula]
	\label{thm: kac-rice formula}
	For a centered Gaussian random field \(\rf\) satisfying our smoothness
	Assumption~\ref{assmpt: smoothness assumption}, the intensity measure of the
	critical points \(\pp\) is given by
	\begin{align*}
		&\intensityMeas(\Vol,L)
		=\E[\pp(\Vol, L)]\\
		&= \int_\Vol \E\left[
			\left|\det(\nabla^2 \rf(t))\right|
			\ind_L(\rf(t), \nabla^2\rf(t))
			\bigm| \nabla\rf(t) = 0 
		\right] \density_{\nabla\rf(t)}(0)dt
		\\
		\overset{\text{stationary}}&=
		|\Vol|\ \underbrace{\E\left[
			|\det(\nabla^2 \rf(0))|
			\ind_L(\rf(0), \nabla^2\rf(0))
		\right] \density_{\nabla\rf(0)}(0)}_{=:\markIntMeas(L)}
	\end{align*}
	where \(\density_{\nabla\rf(t)}\) is the density of \(\nabla\rf(t)\). The
	second equality is only true when \(\rf\) is stationary.
\end{theorem}

\begin{proof}
	This is an immediate result from Corollary~\ref{cor: gaussian kac-rice}
	\parencite[Corollary~11.2.2]{adlerRandomFieldsGeometry2007}. More precisely
	we define
	\begin{align*}
		f(t) &:= \nabla \rf(t)\\
		g(t) &:= (\rf(t), \nabla^2 \rf(t))
	\end{align*}
	We now simply need to check that the covariance function of the components of
	\(\partial_j f^i(t) = \partial_{ji} \rf(t)\) and \(g^i\) (which are
	also just the second derivatives and \(\rf\) itself) satisfy the smoothness
	requirement of the corollary. As smoothness of the second derivative implies
	smoothness of \(\rf\), we only need to check the requirement for \(\partial_{ji} \rf\).
	In other words, we need
	\[
		\max_{i,j}|\partial_{x_i x_j, y_i y_j}\C(t,t)
		+ \partial_{x_i x_j, y_i y_j}\C(s,s)
		-2\partial_{x_i x_j, y_i y_j}\C(s,t)| \le K |\ln(|t-s|)|^{-(1+\alpha)}
	\]
	But that is \eqref{eq: sufficient for continuous second derivative} in
	Assumption~\ref{assmpt: smoothness assumption} written in terms of
	covariances. By application
	of Corollary~\ref{cor: gaussian kac-rice}, we therefore get
	\begin{align*}
		\intensityMeas(\Vol, L) 
		&= \E[\pp(\Vol, L)]
		= \E[\level(f, g: \Vol, L)]\\
		&= \int_\Vol \E\left[
			|\det \nabla f(t)| \ind_L(\rf(t), \nabla^2\rf(t)) \mid f(t) = 0
		\right] \density_{\nabla\rf(t)}(0)dt.
	\end{align*}
	
	
	The second inequality follows as \(\nabla\rf(t)\) is independent of
	\((\rf(t),\nabla^2\rf(t))\) due to Corollary~\ref{cor: uncorr leads to indep
	in gaussian case} which turns the conditional expectation into a normal
	expectation. Lastly replace all \(t\) with \(0\) due to stationarity.
\end{proof}
\begin{remark}
	As can be seen in Theorem~\ref{thm: general kac-rice} \parencite[Theorem
	11.2.1]{adlerRandomFieldsGeometry2007}, \(\rf\) does not necessarily
	need to be Gaussian. We just need it and its derivatives to satisfy a
	list of properties which are satisfied for Gaussian process with
	smooth enough covariance functions.
\end{remark}

Due to stationarity, we have completely dropped the dependence of \(\rf\) on
\(t\) in Theorem~\ref{thm: kac-rice formula} to define the intensity measure
of marks \(\markIntMeas\). To reduce notational clutter, we will from now on write
\[
	(\rf,\nabla\rf, H):= (Z(0),\nabla\rf(0), \nabla^2\rf(0)),
\]
i.e. the intensity measure of marks is
\begin{equation}\label{eq: intensity meas of marks}
	\markIntMeas(L) = \E[|\det(H)| \ind_L(\rf,H)] \varphi_{\nabla \rf}(0)
\end{equation}

\subsection{Mark Distribution}

For a random critical point we would like to talk about its mark
distribution -- think ``height distribution of minimas''. If \(\pp\) is a finite measure (finite set), then picking a random
critical point uniformly is equivalent to sampling
\((T,\Mark)=(T,\rf_T,H_T)\sim\pp\).  Where we rescale \(\pp\) to be a
probability measure. But of course \(\pp\) will generally not be finite over
\(\Vol\), as its intensity is proportional to the lebesgue measure. So let us
look at the point process restricted to a compact \(\Vol\) first. Asking for the
probability of \(\Mark\) in \(L\) then results in
\[
	\Pr_\Mark(L)
	=\frac{\E[\pp(\Vol, L)]}{\E[\pp(\Vol, \real^{1+\dims^2})]}
	\overset{\text{Thm.~}\ref{thm: kac-rice formula}}=
	\frac{\bcancel{|\Vol|}\markIntMeas(L)}{\bcancel{|\Vol|}\markIntMeas(\real^{1+\dims^2})}
	= \frac{\markIntMeas(L)}{\markIntMeas(\real^{1+\dims^2})}
\]
As this distribution is independent from \(\Vol\), we simply call
\(\Pr_\Mark\) the \textbf{distribution of marks} on the unrestricted point
process of critical points.  If we define the
\textbf{intensity} as
\[
	\intensity
	:= \markIntMeas(\real^{1+\dims^2})
	\overset{\eqref{eq: intensity meas of marks}}=
	\E[|\det(H)|]\varphi_\rf(0),
\]
we can write our intensity measure \(\intensityMeas\) as
\[
	\intensityMeas(\Vol, L) 
	\overset{\text{Thm.~}\ref{thm: kac-rice formula}}=
	|\Vol|\markIntMeas(L)
	= \intensity |\Vol| \Pr_\Mark(L).
\]
In other words, the intensity \(\intensity\) determines how many critical points
there are in \(\Vol\) on average\footnote{
	With Campbell's theorem we also have
	\[
		\E\left[\sum_{[x,m]\in\pp} f(x,m)\right]
		=\E\left[\int f(x,m) \pp(dx,dm)\right]
		\overset{\text{Cam.}}= \intensity \int \int f(x,m) \Pr_M(dm) dx.
	\]
}.

\subsubsection{Decomposing Critical Points by Index}

Let the index of \(H\) be the share of negative eigenvalues\footnote{
	Since the eigenvalues are continuous in the entries of the matrix, the
	number of negative eigenvalues is a measurable function in \(H\).
}
\[
	\negEV(H) := \negEV(\evH) := \frac1\dims \sum_{k=1}^\dims \ind_{\evH_i < 0},
\]
where \(\evH=(\evH_1,\dots,\evH_\dims)\) are the eigenvalues of a symmetric
matrix \(H\).
The sets \(\negEV^{-1}(\tfrac{k}\dims)\) for \(k\in\{0,\dots,\dims\}\) are
disjoint and its union includes all symmetric matrices. And since the hessian of
a random field is almost surely symmetric, we can further decompose our intensity measure
of critical points into
\[
	\intensityMeas(\Vol, L)
	= \intensity |\Vol| \Pr_\Mark(L)
	= \intensity |\Vol| \sum_{k=0}^\dims
	\underbrace{\Pr(\negEV(H_T) = \tfrac{k}\dims)}_{
		\text{relative intensity}
	}\Pr(M\in L\mid \negEV(H_T)=\tfrac{k}\dims),
\]
where
\[
	\Pr(\negEV(H_T) = \alpha)
	= \Pr_M(\real\times\negEV^{-1}(\alpha))
	= \frac{\markIntMeas(\real\times\negEV^{-1}(\alpha))}{\intensity}
	\overset{\eqref{eq: intensity meas of marks}}=
	\frac{\E[|\det(H)|\ind_{\negEV(H)=\alpha}]}{\E[|\det(H)|]},
\]
and
\begin{equation}
	\label{eq: index conditional distribution of marks}
	\begin{aligned}
	\Pr(\Mark\in L\mid \negEV(H_T) = \alpha)
	&= \frac{
		\Pr_\Mark(L \cap \real\times\negEV^{-1}(\alpha))
	}{\Pr_\Mark(\real\times\negEV^{-1}(\alpha))}
	= \frac{
		\markIntMeas(L \cap \real\times\negEV^{-1}(\alpha))
	}{\markIntMeas(\real\times\negEV^{-1}(\alpha))}\\
	\overset{\eqref{eq: intensity meas of marks}}&= \frac{
		\E[|\det(H)|\ind_{\negEV(H) = \alpha}\ind_{(Z,H)\in L}]
	}{\E[|\det(H)|\ind_{\negEV(H) = \alpha}]}.
	\end{aligned}
\end{equation}
Similarly we could define the point process of critical points with index
\(\alpha\) as
\[
	\pp^\alpha(\Vol, L)
	:= \pp(\Vol, L\cap \real\times\negEV^{-1}(\alpha)).
\]
Resulting in the decomposition \(\pp = \sum_{k=0}^\dims \pp^{\frac{k}\dims}\).

\subsubsection{Height Distribution of Minimas}
\label{subsubsec: height distriubtion of minimas}

We are especially interested in minima of course. And \(\pp^0\) is the point
process of minima\footnote{
	Technically ``no negative eigenvalues'' is only a necessary criterion for a
	minima. But as eigenvalues will almost surely never be zero, we can
	ignore this issue. If one wanted to convince themselves that this is
	no issue, they could use ``\(\le\)'' in the definition of \(\negEV\) to
	obtain \(\tilde{\negEV}\). For the height distribution of minima one
	could then use a sandwich argument on
	\[
		\Pr(\rf_T\in A, \negEV(H_T) = 0)
		\ge \Pr(\rf_T\in A, T\text{ is min})
		\ge \Pr(\rf_T\in A, \tilde{\negEV}(H_T) = 0),
	\]
	with \((T,\rf_T,H_T)\sim \pp\).
	By doing the same steps with \(\tilde{\negEV}\) as with
	\(\negEV\), one can convince themselves about equality of the first and
	last term. This then directly translates to the conditionals.
}.
But much more interesting than the point process of minima is
their height distribution. So we select \(\alpha=0\) and \(L=A\times
\real^{\dims\times\dims}\) in \eqref{eq: index conditional distribution of marks}
to obtain
\[
	\Pr(Z_T \in A \mid \underbrace{\negEV(H_T)=0}_{T\text{ is min}})
	= \frac{\E[|\det(H)|\ind_{\negEV(H)=0}\ind_A(Z)]}{\E[|\det(H)|\ind_{\negEV(H)=0}]}
\]
