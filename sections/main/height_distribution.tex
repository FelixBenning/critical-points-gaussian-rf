\section{Height Distribution of Minima}

Let \((T,\rf(T),\nabla^2\rf(T))\sim\crit\) be a random critical point.
Of particular interest, is the height distribution of minima
\[\begin{aligned}
	\Pr(\rf(T)\in A \mid T\text{ is minimum})
	&=\Pr(\rf(T)\in A \mid \negEV[\nabla^2\rf(T)]=0)\\
	&= \frac{\Pr_{(\rf(T), \nabla^2\rf(T))}(A\times \negEV^{-1}(0))}{\Pr(\negEV[\nabla^2\rf(T)]=0)}\\
	&= \int_A \underbrace{
		\frac{\int_{\negEV^{-1}(0)} \Omega(z,h) dh}{\Pr(T\text{ is min})}
	}_{=: \phi(z)} dz.
\end{aligned}\]
It turns out to be absolutely continuous with density \(\phi\). Unfortunately
our term for \(\phi\) looks horrible. While \(\Omega\) itself might be
calculable, the set \(\negEV^{-1}(0)\) is anything but pretty. We are therefore
going to switch to the eigenvalues \(\lambda_1,\dots,\lambda_\dimension\) of
the random matrix \(H\). We have
\[
	\det(H) = \prod_{i=1}^\dimension \Lambda_i
	\quad\text{and}\quad
	\ind_{\negEV(H) = 0} = \prod_{i=1}^\dimension \ind_{\Lambda_i\ge 0}
\]
this leads to 
\[\begin{aligned}
	&\Pr_{(\rf(T), \nabla^2\rf(T))}(A\times \negEV^{-1}(0))\\
	&\overset{\text{def. }\Omega}= c\int_{A\times\negEV^{-1}(0)} |\det(h)| p(z,h)dz dh\\
	&\overset{\text{def. } p}= c\E[\ind_{A}(Z)\ind_{\negEV^{-1}(0)}(H)|\det(H)|]\\
	&= c\E[\ind_{A}(Z)\prod_{i=1}^\dimension |\Lambda_i| \ind_{\Lambda_i \ge 0}]\\
	&= \int_A c\underbrace{
		\E\Big[
			\int \prod_{i=1}^\dimension |\Lambda_i| \ind_{\Lambda_i\ge 0}
			\Bigm| Z=z
		\Big]
		\varphi_Z(z)
	}_{\sim\phi(z)}
	dz,
\end{aligned}\]
where \(\varphi_Z(z)\) is the marginal density of \(Z=Z(0)\). Now recall that we
derived the conditional distribution in Subsection~\ref{subsec: Lambda|rf}. And
it were simply GOE eigenvalues with an independent shift. So the height
distribution of minima is therefore up to constants
\[\begin{aligned}
	&\phi(z)\\
	\overset{\eqref{eq: expectation f(Lambda) | Z}}&\sim
	\E\Bigg[
		\prod_{i=1}^\dimension \left|
			\lambda_i  + \epsilon + \tfrac{\sqC'(0)}{\sqC(0)}z
		\right| \ind_{\lambda_i\ge - \left(\epsilon + \tfrac{\sqC'(0)}{\sqC(0)}z\right)}
	\Bigg]
	\varphi_\rf(z)\\
	&\sim \int \E\Bigg[
		\prod_{i=1}^\dimension \left|
			\lambda_i  + t + \tfrac{\sqC'(0)}{\sqC(0)}z
		\right|
		\ind\{\min_{1\le j\le\dimension}\lambda_j\ge - \left(t + \tfrac{\sqC'(0)}{\sqC(0)}z\right)\}
	\Bigg]
	\varphi_\epsilon(t)dt
	\varphi_\rf(z),
\end{aligned}\]
where \(\varphi_\epsilon\) is the density of \(\normal\bigl(0,
\sqC''(0)-\frac{\sqC'(0)^2}{\sqC(0)}\bigr)\).

The difficult part is
\[
	f_\dimension(b):=\E\Bigg[
		\prod_{i=1}^\dimension \left|
			\lambda_i - b
		\right|
		\ind\{\min_{1\le j\le\dimension}\lambda_j\ge b\}
	\Bigg]
\]
afterwards we simply get
\[\begin{aligned}
	\phi(z)
	&\sim \varphi_\rf(z)
	\int f_\dimension\bigl(\tfrac{-\sqC'(0)}{\sqC(0)}z - t\bigr)
	\varphi_\epsilon(t)dt\\
	&= \varphi_\rf(z)
	(f_\dimension * \varphi_\epsilon)\bigl(\tfrac{-\sqC'(0)}{\sqC(0)}z\bigr),
\end{aligned}\]
where \(f*g\) denotes the convolution of \(f\) and \(g\).

% Let us first have a look at
% \[\begin{aligned}
% 	\prod_{i=1}^\dimension \ind_{\lambda_i\ge - b}
% 	&= \ind_{\forall i : \lambda_i \ge -b}
% 	= \ind_{\neg(\exists i : \lambda_i < -b)}\\
% 	&= \ind\Big\{
% 		\Big(
% 			\frac1\dimension\sum_{i=1}^\dimension
% 			\ind_{
% 				\frac{\lambda_i}{\sqrt{\dimension\sqC''(0)}} < -\frac{b}{\sqrt{\dimension\sqC''(0)}}
% 			}
% 		\Big) \le 0
% 	\Big\}\\
% 	&= \ind\Big\{
% 		\int \ind_{x < -\frac{b}{\sqrt{\dimension\sqC''(0)}}}
% 		L_\dimension(dx) \le 0
% 	\Big\}\\
% 	&= \ind\Big\{
% 		L_\dimension\big( (-\infty, -\tfrac{b}{\sqrt{\dimension\sqC''(0)}})\big)
% 		\le 0
% 	\Big\},
% \end{aligned}\]
% where
% \[
% 	L_\dimension
% 	:= \frac1\dimension\sum_{i=1}^\dimension
% 	\delta_{\frac{\lambda_i}{\sqrt{\dimension\sqC''(0)}}}
% \]
% is the empirical spectral distribution of the Wigner matrix
% \(\frac1{\sqrt{\dimension\sqC''(0)}}\tilde{H}\). To stabilize the distribution
% of minima, we are going to look at the scaled process \(\rf^s :=
% \frac1{\sqrt{\dimension}}\rf^u\) with covariance function \(\sqC_s(h) =
% \frac{\sqC_u(h)}\dimension\) of an unscaled process \(\rf^u\) with fixed
% covariance function \(\sqC_u\).
% We then have \(\sqC_s''(h)=\frac{\sqC_u''(h)}{\dimension}\). So if we apply
% the results we got so far to \(\rf^s\), we get
% \[
% 	\prod_{i=1}^\dimension \lambda^s_i \ge -b
% 	= 
% 	\ind\Big\{
% 		L^s_\dimension\big( (-\infty, -\tfrac{b}{\sqrt{\sqC_u''(0)}})\big)
% 		\le 0
% 	\Big\},
% \]
% By Wigner's theorem \(L^s_\dimension\) converges almost surely against the
% semicircle law \(\sigma\) with density
% \[
% 	\varphi^\sigma(x) = \frac1{2\pi}\sqrt{4-x^2}\ind_{|x|\le 2}
% \]
% Since \(\sigma\) is absolutely continuous, every open interval is a continuity
% set, so by the Portmanteau theorem we have almost surely
% \[\begin{aligned}
% 	X_\dimension(b)&:=L^s_\dimension\Bigl(
% 		\bigl(-\infty, -\frac{b}{\sqrt{\sqC_u''(0)}}\bigr)
% 	\Bigr)\\
% 	&\to \sigma\Bigl(
% 		\bigl(-\infty, -\frac{b}{\sqrt{\sqC_u''(0)}}\bigr)
% 	\Bigr)\\
% 	&= \begin{cases}
% 		0 & 2\sqC_u''(0) \le b\\
% 	 	\frac1{2\pi}\int_{-2}^{-b/\sqrt{\sqC_u''(0)}} \sqrt{4-x^2}dx & b\in \sqC_u''(0)(-2,2)\\
% 		1 & b\le -2\sqC''(0)
% 	\end{cases}
% \end{aligned}\]
