\section{Getting to the Point Process}

The set of critical points of a random field, is surprise, surprise: random.
So how would one formalize a random set in probability theory? It turns
out that this becomes much easier if we translate our random set to a random
measure. Let us consider the random set \(\Phi\)
\[
	\Phi = \{ X_1, \dots, X_K\}
\]
where \(X_1,\dots, X_K\in\real^\dimension\) are the random points of this set. Then with some
abuse of notation, we can define the measure
\[
	\Phi :
	\begin{cases}
		\borel(\real^\dimension) \to \real \\
		A \mapsto \Phi(A):= \sum_{X\in \Phi}\delta_X(A) 
	\end{cases}
\]
Where \(\delta_X(A)=\ind_A(X)\) is the dirac measure. This measure \(\Phi\)
essentially counts the number of points in the intersection \(\Phi\cap A\).
Testing with different \(A\) allows us to fully determine, what
the original points in \(\Phi\) were. So in this sense, the measure is
equivalent to the set. And it turns out that it is much more convenient
formally, to define random measures, than it is to define random sets.
So let us introduce some nomenclature for these specific measures.

We want our point-counting measure \(\mu\) to be
\begin{itemize}
	\item \textbf{locally finite}, i.e. \(\mu(A) < \infty\) for all bounded
	\(A\in\borel(\real^\dimension)\),
	\item a \textbf{counting measure}, i.e. it should be discrete,
	\item \textbf{simple}, i.e. \(\mu(\{x\})\in\{0,1\}\), if we do not want
	to allow duplicate points.
\end{itemize}
Let \(\locFiniteMeasure\) be the \textbf{set of all locally finite counting measures}
on \(\real^\dimension\) and \(\locFiniteMeasAlg\) the smallest \(\sigma\)-algebra
such that
\[
	\pi_B : \begin{cases}
		\locFiniteMeasure \to \borel(\real)\\
		\mu \mapsto \mu(B)
	\end{cases}
\]
is measurable for every \(B\in\borel(\real^\dimension)\). I.e.
\[
	\locFiniteMeasAlg := \sigma(\pi_B : B\in\borel(\real^\dimension))
\]

\begin{definition}[Point Process]
	A \textbf{point process} is a measurable mapping \(\Phi:\Omega\to\locFiniteMeasure\)
	from a probability space \((\Omega, \mathcal{A}, \Pr)\) into the set of
	locally finite counting measures. Due to the definition of \(\locFiniteMeasAlg\)
	\[
		\Phi(B) := \pi_B\circ \Phi : \Omega \to \real
	\]
	is a random variable for any \(B\in\borel(\real^\dimension)\). If
	\[
		\Lambda:\begin{cases}
			\borel(\real^\dimension)\to [0,\infty)\\
			B \mapsto \E[\Phi(B)]
		\end{cases}
	\]
	is locally finite, then it is called the \textbf{intensity measure}
	(expectation measure) of \(\Phi\). The intensity measure essentially tells
	us, how many points we should expect in some set \(B\) on average.
	If \(\Lambda(dx) = \lambda dx\) for some
	constant \(\lambda\in\real\), then \(\lambda\) is called the
	\textbf{intensity} of \(\Phi\).
	A point process is called \textbf{simple}, if \(\Phi(\omega)\) is simple
	for every \(\omega\in\Omega\).
\end{definition}

Often our random set of points has a certain structure. E.g. we have
random points in space \(X_1,\dots, X_K\) with marks \(M_1,\dots, M_k\) which
might for example represent the value of some random field \(\rf\) at our
points in space, i.e. \(M_i=\rf(X_i)\). Then our random graph 
\[
	\Phi = \{ (X_1, M_1), \dots, (X_K, M_K) \}
\]
is just a point process in higher dimensional space, with points \((X_i, M_i)\).
But of course not every point process can be interpreted as a point with a mark.
So we define the special case

\begin{definition}[Marked Point Process]
	A point process \(\Phi\) on \(\locFiniteMeasure[\dimension+n]\) is called
	\textbf{marked point process}, if \(\Phi_p = \Phi(\cdot \times \real^n)\)
	is a point process on \(\locFiniteMeasure\).

	A marked point process is called \textbf{stationary}, if for all
	\(h\in\real^\dimension\), \(A_i\in\borel(\real^\dimension)\),
	\(M_i\in\borel(\real^n)\) and \(B_i\subseteq\nat_0\) we have
	\[
		\Pr(\Phi(A_i\times M_i) \in B_i, i=1,2,\dots)
		= \Pr(\Phi(A_i+h \times M_i)\in B_i, i=1,2,\dots)
	\]
\end{definition}

Notice that for a marked point process to be stationary, we only shift the
spacial part of the point. This stands in contrast to a stationarity definition
for ordinary point processes. This is the main reason we introduce this
additional machinery. In our case we are interested in the height (mark) of
our random field at our critical points. And the following Lemma will allow us
to recover the distribution of marks on these critical points

\begin{lemma}[Campbell]\label{lem: Campbell}
	For a stationary marked point process \(\Phi\) with intensity \(\Lambda\),
	we have	
	\[
		\Lambda(\Vol \times L) = \lambda(L) |\Vol|
	\]
	where \(|\Vol|\) is the lebesgue measure of \(\Vol\). We call
	\[
		\Pr_M(L) = \frac{\lambda(L)}{\lambda(\real^n)}
	\]
	the \textbf{distribution of marks}, because we have for integrable \(f\)
	(Campbell)
	\[
		\E\left[\int f(x,m) \Phi(dx,dm)\right]
		= \lambda \int \int f(x,m) \Pr_M(dm) dx
	\]
\end{lemma}
\begin{proof}
	\fxwarning{TODO: Fakt in 3.5 räumliche statistik skript schlather}
\end{proof}
