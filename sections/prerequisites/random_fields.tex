\section{Random Fields}

\begin{definition}[Random Field]
	A collection of random variables \((\rf(t))_{t\in\real^\dims}\) is called
	\textbf{random field} over \(\real^\dims\). The \textbf{covariance function}
	is defined as
	\begin{equation*}
		\C(x,y) = \Cov(\rf(x), \rf(y)).
	\end{equation*}
	A random field is called \textbf{Gaussian random field}, if all finite
	dimensional marginals are multivariate Gaussian.
	A random field is called \textbf{centered}, if for all \(t\in\real^\dims\)
	\[
		\E[\rf(t)] = 0.
	\]
	A random field is called \textbf{strictly stationary}, if for all
	\(h\in\real^\dims\) we have the following equality in distribution
	\[
		(\rf(t+h))_{t\in\real^\dims} \overset{d}= (\rf(t))_{t\in\real^\dims}.
	\]
	In particular this implies	\textbf{(weak) stationarity}, i.e.
	\begin{equation*}
		\C(x,y) = \C(x-y)\qquad \text{and} \qquad \E[\rf(t)]=\mu.
	\end{equation*}
	For Gaussian random fields, strict and weak stationarity coincides.
	A random field is called \textbf{isotropic} (rotation invariant) if	
	\begin{equation*}
		\C(x,y) = \sqC\left(\frac{\|x-y\|^2}{2}\right).
	\end{equation*}
\end{definition}

\subsection{Differentiability}

\begin{lemma}[Covariance and Derivative]
	\label{lem: covariance of derivative}
	Assume the partial derivative \(\partial_i\rf(x)\) to exists almost surely and
	\(x\mapsto\Var(\partial_i\rf(x))\) to be continuous in some neighborhood of
	\(x\). Then we have
	\begin{equation*}
		\Cov(\partial_i\rf(x), \rf(y)) = \partial_{x_i} \C(x,y).
	\end{equation*}
	Since \(\partial_i\rf(x)\) is just a random field again, we can iterate
	on this idea to get the covariance between \(\partial_i\rf(x)\) and
	\(\partial_j\rf(y)\) or higher order derivatives \(\partial_{ij}\rf(x)\).
\end{lemma}
\begin{proof}
	Assume without loss of generality \(Z\) is centered. To apply
	Theorem~\ref{thm: swap integration and differentiation} we need
	\(x\mapsto\E[|\partial_i\rf(x)\rf(y)|]\) to be integrable in a local neighborhood.
	But by the Hölder inequality, we have
	\[
		\E[|\partial_i\rf(x)\rf(y)|]
		\le \E[\partial_i\rf(x)^2]^{\tfrac12} 
		\underbrace{\E[\rf(y)^2]^{\tfrac12}}_{=\sqrt{\C(y,y)}}
	\]
	and we assumed \(x\mapsto\E[\partial_i\rf(x)^2]\) to be continuous. It is therefore
	locally integrable and we have
	\begin{align*}
		\Cov(\partial_i \rf(x), \rf(y))
		&= \int \partial_{x_i}\rf(x,\omega) \rf(y,\omega) d\Pr(\omega)
		\overset{\ref{thm: swap integration and differentiation}}= \partial_{x_i} \int \rf(x,\omega) \rf(y,\omega) d\Pr(\omega)\\
		&= \partial_{x_i} \C(x,y).
		\qedhere
	\end{align*}
\end{proof}
Since we want
\[
	\Cov(\partial_i \rf(x), \partial_j\rf(y)) = \partial_{x_i}\partial_{y_i}\C(x,y)
	\overset{\text{stationary case}}= -\partial_{ij}\C(x-y),
\]
it is natural that the existence of this second order derivative of \(\C\) is
necessary for the existence of directional derivatives of \(\rf\). It turns out
that it is also almost sufficient. If \(\C\) is twice differentiable, then
\[
	\Delta_v^n \rf(x)	:= \frac{\rf(x+\frac{v}n)-\rf(x-\frac{v}n)}{2/n}
\]
is an \(L^2\) cauchy sequence and its \(L^2\) limit therefore exists and is
almost surely unique. In particular we can now define
\[
	\partial_i\rf(x):=\lim_{n\to\infty} \Delta_{e_i}^n\rf(x)
\]
in the \(L^2\) sense. Whenever \(\rf\) is Gaussian \((\Delta_v^n \rf(x),
\rf(y))\) is multivariate Gaussian, and therefore also its limit.

But we want almost sure point-wise limits, not \(L^2\) limits. To recover this
property, we can use a trick from 
\textcite[Sec.~1.4.2]{adlerRandomFieldsGeometry2007}. We define the random field
\[
	F: \begin{cases}
		\real^\dims \times \real^\dims \times \real \to \real\\
		(x,v,h)	\mapsto \frac{\rf(x+hv)-\rf(x-hv)}{2h}
	\end{cases}
\]
where \(\Delta_i^n Z(x) = F(x,e_i,\tfrac1n)\). Technically \(F(x,v,0)\) is not
well defined right now. In its place, we insert the appropriate directional
derivative (which exists in the \(L^2\) sense). If this insertion leaves \(F\)
to be almost surely continuous, then we have almost surely
\[
	\Delta_i^n Z(x)=F(x,e_i,\tfrac1n)\to F(x, e_i, 0)= \partial_i Z(x).
\]
we obtain for free, that \(x\mapsto \partial_i Z(x) = F(x,e_i,0)\) is almost
surely continuous. So we only need a condition for random fields to be almost
surely continuous and simply apply this to \(F\) instead of \(\rf\). Higher
order derivatives work similarly. For details see
\textcite[Sec.~1.4.2]{adlerRandomFieldsGeometry2007}.

To count critical points with restrictions on the second derivative, we need
the second derivative to exist. And continuity would be nice. The following
assumption is sufficient for both \parencite*[cf.
Theorem~1.4.2]{adlerRandomFieldsGeometry2007}

\begin{assumption}[Smooth Random Field]\label{assmpt: smoothness assumption}
	\fxwarning{Copied wrong theorem so far}
	A sufficient condition for smooth second derivatives \(\partial_{ij}	\rf\)
	of a random field, is
	\begin{equation}\label{eq: sufficient for continuous second derivative}
		\max_{i,j}\E[|\partial_{ij}\rf(t) - \partial_{ij}\rf(s)|^2]
		\le K |\ln(|t-s|)|^{-(1+\alpha)}
	\end{equation}
	for some \(K>0\), \(\alpha>0\) for all \(|t-s|\) small enough.
	In the stationary case this condition simplifies to
	\[
		\max_{i,j}|\partial_{ii jj}\C(0)
		-\partial_{ii jj}\C(h)| \le \frac{K}2 |\ln(|h|)|^{-(1+\alpha)}
	\]
	for \(h\in\real^\dims\) with \(|h|\) small enough.
\end{assumption}


\begin{remark}\label{rem: covariance uncorrelated}
	For stationary random fields \(\C(x,y)=\C(x-y)\), the covariance
	function \(\C\) is symmetric, i.e. \(\C(h)=\C(-h)\)
	and therefore the derivative antisymmetric, i.e.
	\(\partial_i\C(-h)=-\partial_i\C(h)\). In particular
	\begin{equation*}
		\Cov(\nabla\rf(x), \rf(x)) = \nabla \C(0)=0.
	\end{equation*}
\end{remark}
\begin{corollary}[Gaussian Case] \label{cor: uncorr leads to indep in gaussian case}
	If \(\rf\) is a stationary gaussian random field, \(\rf(x)\) and
	\(\nabla\rf(x)\) are independent multivariate gaussian for every \(x\).
\end{corollary}
